\chapter{The Frobenius element}
Throughout this chapter $K/\QQ$ is a Galois extension with Galois group $G$,
$p$ is an \emph{unramified} rational prime in $K$, and $\kp$ is a prime above it.
Picture:
\begin{center}
\begin{tikzcd}
	K \ar[d, dash]
		& \supset
		& \OO_K \ar[d, dash]
		& \kp \ar[d, dash]
		& \OO_K/\kp \cong \FF_{p^f} \ar[d, dash] \\
	\QQ & \supset & \ZZ & (p) & \FF_p
\end{tikzcd}
\end{center}
If $p$ is unramified, then one can show there
is a unique $\sigma \in \Gal(L/K)$ such that
$\sigma(\alpha) \equiv \alpha^p \pmod{\kp}$ for every prime $p$.

\section{Frobenius elements}
\prototype{$\Frob_\kp$ in $\ZZ[i]$ depends on $p \pmod 4$.}

Here is the theorem statement again:
\begin{theorem}[The Frobenius element]
	Assume $K/\QQ$ is Galois with Galois group $G$.
	Let $p$ be a rational prime unramified in $K$, and $\kp$ a prime above it.
	There is a \emph{unique} element $\Frob_\kp \in G$
	with the property that
	\[ \Frob_\kp(\alpha) \equiv \alpha^{p} \pmod{\kp}. \]
	It is called the \vocab{Frobenius element} at $\kp$, and has order $f$.
\end{theorem}
The \emph{uniqueness} part is pretty important:
it allows us to show that a given $\sigma \in \Gal(L/K)$
is the Frobenius element by just observing that it satisfies
the above functional equation.

Let's see an example of this:
\begin{example}[Frobenius elements of the Gaussian integers]
	Let's actually compute some Frobenius elements for $K = \QQ(i)$,
	which has $\OO_K = \ZZ[i]$.
	This is a Galois extension, with $G = \Zm2$,
	corresponding to the identity and complex conjugation.

	If $p$ is an odd prime with $\kp$ above it,
	then $\Frob_\kp$ is the unique element such that
	\[ (a+bi)^p \equiv \Frob_\kp(a+bi) \pmod{\kp} \]
	in $\ZZ[i]$. In particular,
	\[ \Frob_\kp(i) = i^p =
		\begin{cases}
			i & p \equiv 1 \pmod 4 \\
			-i & p \equiv 3 \pmod 4.
		\end{cases}
	\]
	From this we see that $\Frob_\kp$ is the identity when $p \equiv 1 \pmod 4$
	and $\Frob_\kp$ is complex conjugation when $p \equiv 3 \pmod 4$.
\end{example}
Note that we really only needed to compute $\Frob_\kp$ on $i$.
If this seems too good to be true,
a philosophical reason is ``freshman's dream''
where $(x+y)^p \equiv x^p + y^p \pmod{p}$ (and hence mod $\kp$).
So if $\sigma$ satisfies the functional equation on generators,
it satisfies the functional equation everywhere.

We also have an important lemma:
\begin{lemma}
	[Order of the Frobenius element]
	Let $\Frob_\kp$ be a Frobenius element from an extension $K/\QQ$.
	Then the order of $\Frob_\kp$ is equal to the inertial degree $f_\kp$.
	In particular, $(p)$ splits completely in $\OO_K$
	if and only if $\Frob_\kp = \id$.
\end{lemma}
\begin{exercise}
	Prove this lemma as by using the fact that $\OO_K / \kp$
	is the finite field of order $f_\kp$,
	and the Frobenius element is just $x \mapsto x^p$ on this field.
\end{exercise}

Let us now prove the main theorem.
This will only make sense in the context of decomposition groups,
so readers which skipped that part should omit this proof.
\begin{proof}
	[Proof of existence of Frobenius element]
	The entire theorem is just a rephrasing of the fact
	that the map $\theta$ defined in the last section
	is an isomorphism when $p$ is unramified.
	Picture:
	\begin{center}
		\begin{asy}
			size(12cm);
			filldraw( (-4,-2)--(-4,2)--(1.5,2)--(1.5,-2)--cycle, lightblue+opacity(0.2), black);
			label("$G = \operatorname{Gal}(K/\mathbb Q)$", (-1,2), dir(90));
			dot( (-1.1,-1.2) );
			dot( (-1.4,0.9) );
			dot( (-2,1.4) );
			dot( (-2.7,-0.4) );
			dot( (-3.1,0.2) );
			dot( (-3.4,-1.6) );

			filldraw(scale(0.8,1.8)*unitcircle, lightcyan+opacity(0.4), black);
			label("$D_{\mathfrak p}$", (0.8,2), dir(-90));
			for (real y=-1.5; y<2; ++y) { dot( (0,y) ); }
			label("$\operatorname{Frob}_{\mathfrak p}$", (0,-1.5), dir(90));

			filldraw(shift(5,0)*scale(0.8,1.8)*unitcircle, lightcyan+opacity(0.4), black);
			for (real y=0.5; y<2; ++y) { dot( (5,y) ); }
			dot("$T$", (5,-1.5), dir(45));
			dot("$T^2$", (5,-0.5), dir(45));
			draw( (1,0)--(4,0), Arrows );
			label("$\left<T \mid T^f=1\right>$", (5,1.8), dir(90));

			draw( (0.2,-1.5)--(4.8,-1.5), dashed, EndArrow);
			label("$\theta(\operatorname{Frob}_{\mathfrak p}) = T$", (2.8,-1.5), dir(-90));
			label("$\theta$", (2.5,0), dir(90));
			label("$\cong$", (2.5,0), dir(-90));
		\end{asy}
	\end{center}
	In here we can restrict our attention to $D_\kp$
	since we need to have $\sigma(\alpha) \equiv 0 \pmod \kp$
	when $\alpha \equiv 0 \pmod \kp$.
	Thus we have the isomorphism
	\[ D_\kp \taking\theta \Gal\left( (\OO_K/\kp) / \FF_p \right). \]
	But we already know $\Gal\left( (\OO_K/\kp)/\FF_p \right)$,
	according to the string of isomorphisms
	\[
		\Gal\left( (\OO_K/\kp) / \FF_p \right)
		\cong \Gal\left( \FF_{p^f} / \FF_p \right)
		\cong \left< T = x \mapsto x^p \right>
		\cong \Zc{f} .
	\]
	So the unique such element is the pre-image of $T$ under $\theta$.
\end{proof}


\section{Conjugacy classes}
Now suppose $\kp_1$ and $\kp_2$ are \emph{two} primes above an unramified rational prime $p$.
Then we can define $\Frob_{\kp_1}$ and $\Frob_{\kp_2}$.
Since the Galois group acts transitively,
we can select $\sigma \in \Gal(K/\QQ)$ be such that
\[ \sigma(\kp_1) = \kp_2. \]
We claim that
\[
	\Frob_{\kp_2} = \sigma \circ \Frob_{\kp_1} \circ \sigma\inv.
\]
Note that this is an equation in $G$.
\begin{ques}
	Prove this.
\end{ques}

More generally, for a given unramified rational prime $p$, we obtain:
\begin{theorem}
	[Conjugacy classes in Galois groups]
	The set
	\[ \left\{ \Frob_\kp \mid \kp \text{ above } p \right\} \]
	is one of the conjugacy classes of $G$.
\end{theorem}
\begin{proof}
	We've used the fact that $G = \Gal(K/\QQ)$ is transitive
	to show that $\Frob_{\kp_1}$ and $\Frob_{\kp_2}$ are conjugate
	if they both lie above $p$; hence it's \emph{contained} in some
	conjugacy class.
	So it remains to check that for any $\kp$, $\sigma$,
	we have $\sigma \circ \Frob_\kp \circ \sigma\inv = \Frob_{\kp'}$
	for some $\kp'$. For this, just take $\kp' = \sigma\kp$.
	Hence the set is indeed a conjugacy class.
\end{proof}
%We denote the conjugacy class by the \vocab{Frobenius symbol}
%\[ \left( \frac{K/\QQ}{p} \right). \]

In summary,
\begin{moral}
	$\Frob_{\kp}$ is determined up to conjugation by the prime $p$
	from which $\kp$ arises.
\end{moral}
So even though the Gothic letters look scary, the content of $\Frob_{\kp}$
really just comes from the more friendly-looking rational prime $p$.


\begin{example}
	[Frobenius elements in $\QQ(\cbrt2,\omega)$]
	With those remarks, here is a more involved example of a Frobenius map.
	Let $K = \QQ(\cbrt2, \omega)$ be the splitting field of
	\[ t^3-2 = (t-\cbrt2)(t-\omega\cbrt2)(t-\omega^2\cbrt2). \]
	Thus $K/\QQ$ is Galois.
	We've seen in an earlier example that
	\[ \OO_K = \ZZ[\eps] \quad\text{where}\quad \eps \text { is a root of } t^6+3t^5-5t^3+3t+1. \]

	Let's consider the prime $5$ which factors (trust me here) as
	\[ (5) = (5, \eps^2+\eps+2)(5, \eps^2+3\eps+3)(5, \eps^2+4\eps+1)
		= \kp_1 \kp_2 \kp_3. \]
	Note that all the prime ideals have inertial degree $2$.
	Thus $\Frob_{\kp_i}$ will have order $2$ for each $i$.

	Note that
	\[ \Gal(K/\QQ) =
		\text{permutations of } \{\cbrt2,\omega\cbrt2,\omega^2\cbrt2\}
		\cong S_3.  \]
	In this $S_3$ there are $3$ elements of order two:
	fixing one root and swapping the other two.
	These correspond to each of $\Frob_{\kp_1}$, $\Frob{\kp_2}$, $\Frob_{\kp_3}$.

	In conclusion, the conjugacy class
	$\left\{ \Frob_{\kp_1}, \Frob_{\kp_2}, \Frob_{\kp_3} \right\}$
	associated to $(5)$ is the
	cycle type $(\bullet)(\bullet \; \bullet)$ in $S_3$.
\end{example}


\section{Chebotarev density theorem}
Natural question: can we represent every conjugacy class in this way?
In other words, is every element of $G$ equal to $\Frob_\kp$ for some $\kp$?

Miraculously, not only is the answer ``yes'', but in fact it does so in the nicest way possible:
the $\Frob_\kp$'s are ``equally distributed'' when we pick a random $\kp$.
\begin{theorem}
	[Chebotarev density theorem over $\QQ$]
	Let $C$ be a conjugacy class of $G = \Gal(K/\QQ)$.
	The density of (unramified) primes $p$ such that $\{ \Frob_\kp \mid \kp \text{ above } p \} = C$
	%\[ \left( \frac{K/\QQ}{p} \right) = C \]
	is exactly $\left\lvert C \right\rvert / \left\lvert G \right\rvert$.
	In particular, for any $\sigma \in G$ there are infinitely many rational primes $p$
	with $\kp$ above $p$ so that $\Frob_{\kp} = \sigma$.
\end{theorem}

By density, I mean that the proportion of primes $p \le x$ that work
approaches $\frac{\left\lvert C \right\rvert}{\left\lvert G \right\rvert}$ as $x \to \infty$.
Note that I'm throwing out the primes that ramify in $K$.
This is no issue, since the only primes that ramify are those dividing $\Delta_K$,
of which there are only finitely many.

In other words, if I pick a random prime $p$ and look at the resulting conjugacy class,
it's a lot like throwing a dart at $G$:
the probability of hitting any conjugacy class depends just on the size of the class.
\begin{center}
	\begin{asy}
		size(8cm);
		bigbox("$G$");
		pen b = lightcyan + opacity(0.4);
		pen k = black;
		filldraw( (-2.6,2.5)--(0.6,2.5)--(0.6,0.5)--(-2.6,0.5)--cycle, b, k);
		filldraw( (-2.6,-2.5)--(0.6,-2.5)--(0.6,-0.5)--(-2.6,-0.5)--cycle, b, k);
		filldraw( (2,0)--(3.5,0)--(3.5,2.5)--(2,2.5)--cycle, b, k);
		filldraw( (2,-1)--(3.5,-1)--(3.5,-2)--(2,-2)--cycle, b, k);
		for (real x = -2; x < 1; ++x) {
			dot( (x, 1.9) );
			dot( (x, 1.1) );
			dot( (x, -1.9) );
			dot( (x, -1.1) );
		}
		label("$37.5\%$", (-2.6, 0.5), dir(140));
		label("$37.5\%$", (-2.6,-2.5), dir(140));
		label("$C_1$", (-2.6, 2.5), dir(225));
		label("$C_2$", (-2.6, -.5), dir(225));
		dot( (2.75, 2.0) );
		dot( (2.75, 1.25) );
		dot( (2.75, 0.50) );
		dot( (2.75, -1.50) );
		label("$C_3$", (2, 0), dir(-90));
		label("$18.75\%$", (3, 0), dir(-75));
		label("$C_4$", (2, -2), dir(-90));
		label("$6.25\%$", (3, -2), dir(-75));
	\end{asy}
\end{center}

\begin{remark}
Happily, this theorem (and preceding discussion)
also works if we replace $K/\QQ$ with any Galois extension $K/F$;
in that case we replace ``$\kp$ over $p$'' with ``$\kP$ over $\kp$''.
In that case, we use $\Norm(\kp) \le x$ rather than $p \le x$ as the way
to define density.
\end{remark}

\section{Example: Frobenius elements of cyclotomic fields}
Let $q$ be a prime, and consider $L = \QQ(\zeta_q)$,
with $\zeta_q$ a primitive $q$th root of unity.
You should recall from various starred problems that
\begin{itemize}
	\ii $\Delta_L = \pm q^{q-2}$,
	\ii $\OO_L = \ZZ[\zeta_q]$, and
	\ii The map \[ \sigma_n : L \to L \quad\text{by}\quad \zeta_q \mapsto \zeta_q^n \]
	is an automorphism of $L$ whenever $\gcd(n,q)=1$,
	and depends only on $n \pmod q$.
	In other words, the automorphisms of $L/\QQ$ just shuffle around the $q$th roots of unity.
	In fact the Galois group consists exactly of the elements $\{\sigma_n\}$, namely
	\[ \Gal(L/\QQ) = \{ \sigma_n \mid n \not\equiv 0 \pmod q \}. \]
	As a group, \[ \Gal(L/\QQ) = \Zm q \cong \Zcc{q-1}. \]
\end{itemize}
This is surprisingly nice,
because \textbf{elements of $\Gal(L/\QQ)$ look a lot
like Frobenius elements already}.
Specifically:

\begin{lemma}[Cyclotomic Frobenius elements]
	\label{lem:cyclo_frob}
	In the cyclotomic setting $L = \QQ(\zeta_q)$,
	let $p$ be a rational unramified prime
	and $\kp$ above it. Then \[ \Frob_\kp = \sigma_p. \]
\end{lemma}
\begin{proof}
	Observe that $\sigma_p$ satisfies the functional equation
	(check on generators).
	Done by uniqueness.
%	We know $\Frob_\kp(\alpha) \equiv \alpha^p \pmod{\kp}$ by definition,
%	but also that $\Frob_\kp = \sigma_n$ for some $n$
%	We want $n=p$; since $\sigma_n(\zeta_q)^n = \zeta_q^n$ by definition
%	it would be very weird if this wasn't true!
%
%	Given $\zeta_q^n \equiv \zeta_q^p \pmod{\kp}$, it suffices to
%	prove that the $q$th roots of unity are distinct mod $\kp$.
%	Look at the polynomial $F(x) = x^q-1$ in $\ZZ[\zeta_p]/\kp \cong \FF_{p^f}$.
%	Its derivative is \[ F'(x) = qx^{q-1} \not\equiv 0 \pmod{\kp} \]
%	(since $\FF_{p^f}$ has characteristic $p \nmid q$).
%	The only root of $F'$ is zero, hence $F$ has no double roots mod $\kp$.
\end{proof}

\begin{ques}
	Conclude that a rational prime $p$
	splits completely in $\OO_L$ if and only if $p \equiv 1 \pmod q$.
\end{ques}

\section{Frobenius elements behave well with restriction}
Let $L/\QQ$ and $K/\QQ$ be Galois extensions, and consider the setup
\begin{center}
\begin{tikzcd}
	L \ar[d, dash] & \supset
		& \kP \ar[d, dash] \ar[r, dotted]
		& \Frob_{\kP} \in \Gal(L/\QQ)\\
	K \ar[d, dash] & \supset
		& \kp \ar[d, dash] \ar[r, dotted]
		& \Frob_\kp \in \Gal(K/\QQ) \\
	\QQ & \supset
		& (p)
		&
\end{tikzcd}
\end{center}
Here $\kp$ is above $(p)$ and $\kP$ is above $\kp$.
We may define
\[ \Frob_\kp \colon K \to K
	\quad\text{and}\quad
	\Frob_{\kP} \colon L \to L \]
and want to know how these are related.

\begin{theorem}
	[Restrictions of Frobenius elements]
	Assume $L/\QQ$ and $K/\QQ$ are both Galois.
	Let $\kP$ and $\kp$ be unramified as above.
	Then $\Frob_{\kP} \restrict{K} = \Frob_{\kp}$,
	i.e.\ for every $\alpha \in K$,
	\[ \Frob_\kp(\alpha) = \Frob_{\kP}(\alpha). \]
\end{theorem}
%\begin{proof}
%	We know
%	\[ \Frob_{\kP}(\alpha) \equiv \alpha^p \pmod{\kP}
%		\quad \forall \alpha \in \OO_L \]
%	from the definition.
%	\begin{ques}
%		Deduce that
%		\[ \Frob_{\kP}(\alpha) \equiv \alpha^p \pmod{\kp}
%			\quad \forall \alpha \in \OO_K. \]
%		(This is weaker than the previous statement in two ways!)
%	\end{ques}
%	Thus $\Frob_{\kP}$ restricted to $\OO_K$ satisfies the
%	characterizing property of $\Frob_\kp$.
%\end{proof}
\begin{proof}
	TODO: Broken proof. Needs repair.
\end{proof}
In short, the point of this section is that
\begin{moral}
	Frobenius elements upstairs restrict to Frobenius elements downstairs.
\end{moral}

\section{Application: Quadratic reciprocity}
We now aim to prove:
\begin{theorem}
	[Quadratic reciprocity]
	Let $p$ and $q$ be distinct odd primes.
	Then
	\[ \left( \frac pq \right)\left( \frac qp \right)
		= (-1)^{\frac{p-1}{2} \cdot \frac{q-1}{2}}. \]
\end{theorem}
(See, e.g. \cite{ref:holden} for an exposition on quadratic reciprocity,
if you're not familiar with it.)

\subsection{Step 1: Setup}
For this proof, we first define
\[ L = \QQ(\zeta_q) \]
where $\zeta_q$ is a primitive $q$th root of unity.
Then $L/\QQ$ is Galois, with Galois group $G$.
\begin{ques}
	Show that $G$ has a unique subgroup $H$ of index two.
\end{ques}
In fact, we can describe it exactly: viewing $G \cong \Zm q$, we have
\[ H = \left\{ \sigma_n \mid \text{$n$ quadratic residue mod $q$} \right\}. \]
By the fundamental theorem of Galois Theory, there ought to be a degree $2$
extension of $\QQ$ inside $\QQ(\zeta_q)$ (that is, a quadratic field).
Call it $\QQ(\sqrt{q^\ast})$, for $q^\ast$ squarefree:
\begin{center}
\begin{tikzcd}
	L = \QQ(\zeta_q)
		\ar[d, "\frac{q-1}{2}"', dash]
		\ar[r, leftrightarrow]
		& \{1\} \ar[d, dash] \\
	K = \QQ(\sqrt{q^\ast})
		\ar[d, "2"', dash]
		\ar[r, leftrightarrow]
		& H \ar[d, dash] \\
	\QQ \ar[r, leftrightarrow]
		& G
\end{tikzcd}
\end{center}
\begin{exercise}
	Note that if a rational prime $\ell$ ramifies in $K$,
	then it ramifies in $L$.
	Use this to show that
	\[ q^\ast = \pm q \text{ and } q^\ast \equiv 1 \pmod 4. \]
	Together these determine the value of $q^\ast$.
\end{exercise}
(Actually, it is true in general
$\Delta_K$ divides $\Delta_L$ in a tower $L/K/\QQ$.)

\subsection{Step 2: Reformulation}
Now we are going to prove:
\begin{theorem}
	[Quadratic reciprocity, equivalent formulation]
	For distinct odd primes $p$, $q$ we have
	\[ \left( \frac pq \right) = \left( \frac{q^\ast}{p} \right). \]
\end{theorem}
\begin{exercise}
	Using the fact that $\left( \frac{-1}{p} \right) = (-1)^{\frac{p-1}{2}}$,
	show that this is equivalent to quadratic reciprocity as we know it.
\end{exercise}

We look at the rational prime $p$ in $\ZZ$.
Either it splits into two in $K$ or is inert; either way let $\kp$ be a prime factor
in the resulting decomposition (so $\kp$ is either $p \cdot \OO_K$ in the inert case,
or one of the primes in the split case).
Then let ${\kP}$ be above $\kp$.
It could possibly also split in $K$: the picture looks like
\begin{center}
\begin{tikzcd}
	\OO_L = \ZZ[\zeta_q] & \supset
		& {\kP} \ar[r, dotted] & \ZZ[\zeta_p]/{\kP} \cong \FF_{p^f} \\
	\OO_K = \ZZ[\frac{1+\sqrt{q^\ast}}{2}] & \supset
		& \kp \ar[r, dotted] & \FF_p \text{ or } \FF_{p^2} \\
	\ZZ & \supset & (p) \ar[r, dotted]
		& \FF_p
\end{tikzcd}
\end{center}
\begin{ques}
	Why is $p$ not ramified in either $K$ or $L$?
\end{ques}

\subsection{Step 3: Introducing the Frobenius}
Now, we take the Frobenius
\[ \sigma_p = \Frob_{\kP} \in \Gal(L/\QQ). \]
We claim that
\[ \Frob_{\kP} \in H \iff \text{$p$ splits in $K$}. \]
To see this, note that $\Frob_{\kP}$ is in $H$ if and only if it acts
as the identity on $K$.
But $\Frob_{\kP} \restrict{K}$ is $\Frob_\kp$!
So \[ \Frob_{\kP} \in H \iff \Frob_\kp = \id_K. \]
Finally note that $\Frob_\kp$ has order $1$ if $p$ splits
($\kp$ has inertial degree $1$)
and order $2$ if $p$ is inert.
This completes the proof of the claim.

\subsection{Finishing up}
We already know by \Cref{lem:cyclo_frob} that $\Frob_{\kP} = \sigma_p \in H$
if and only if $p$ is a quadratic residue.
On the other hand,
\begin{exercise}
	Show that $p$ splits in $\OO_K = \ZZ[\frac12(1+\sqrt{q^\ast})]$
	if and only if $\left( \frac{q^\ast}{p} \right) = 1$.
	(Use the factoring algorithm. You need the fact that $p \neq 2$ here.)
\end{exercise}
In other words
\[ \left( \frac pq \right) = 1
	\iff \sigma_p \in H \iff \text{$p$ splits in $\ZZ\left[ \tfrac12(1+\sqrt{q^\ast}) \right]$}
	\iff \left( \frac{q^\ast}{p} \right) = 1.
\]
This completes the proof.


\section{Frobenius elements control factorization}
\prototype{$\Frob_\kp$ controlled the splitting of $p$ in the proof of quadratic reciprocity;
the same holds in general.}
In the proof of quadratic reciprocity, we used the fact that Frobenius elements behaved
well with restriction in order to relate the splitting of $p$ with properties of $\Frob_\kp$.

In fact, there is a much stronger statement for
any intermediate field $\QQ \subseteq E \subseteq K$
which works even if $E/\QQ$ is not Galois.
It relies on the notion of a \emph{factorization pattern}.
Here is how it goes.

Set $n = [E:\QQ]$, and let $p$ be a rational prime unramified in $K$.
Then $p$ can be broken in $E$ as
\[ p \cdot \OO_E = \kp_1 \kp_2 \dots \kp_g \]
with inertial degrees $f_1$, \dots, $f_g$:
(these inertial degrees might be different since $E/\QQ$ isn't Galois).
The numbers $f_1 + \dots + f_g = n$ form a partition of the number $n$.
For example, in the quadratic reciprocity proof we had $n = 2$,
with possible partitions $1 + 1$ (if $p$ split) and $2$ (if $p$ was inert).
We call this the \vocab{factorization pattern} of $p$ in $E$.

Next, we introduce a Frobenius $\Frob_{\kP}$ above $(p)$, all the way in $K$;
this is an element of $G = \Gal(K/\QQ)$.
Then let $H$ be the group corresponding to the field $E$.
Diagram:
\begin{center}
\begin{tikzcd}
	K \ar[r, leftrightarrow] \ar[d, dash] & \{1\} \ar[d, dash]
		& \Frob_{\kP} \\
	E \ar[d, dash, "n"'] \ar[r, leftrightarrow] & H \ar[d, dash, "n"]
		& \kp_1 \dots \kp_g \ar[d, dash] & f_1 + \dots + f_g = n \\
	\QQ \ar[r, leftrightarrow] & G & (p)
\end{tikzcd}
\end{center}
Then $\Frob_{\kP}$ induces a \emph{permutation}
of the $n$ left cosets $gH$ by left multiplication
(after all, $\Frob_{\kP}$ is an element of $G$ too!).
Just as with any permutation, we may look at the resulting cycle decomposition,
which has a natural ``cycle structure'': a partition of $n$.
\begin{center}
	\begin{asy}
		size(8cm);
		pen tg = heavyred; // "times g"

		pen pointpen = lightblue;
		pair A = Drawing("g_1H", dir(80), dir(80), pointpen);
		pair B = Drawing("g_2H", A*dir(120), A*dir(120), pointpen);
		pair C = Drawing("g_3H", A*dir(240), A*dir(240), pointpen);
		draw(A--B, dashed + pointpen, EndArrow, Margin(2,2));
		draw(B--C, dashed + pointpen, EndArrow, Margin(2,2));
		draw(C--A, dashed + pointpen, EndArrow, Margin(2,2));
		label("$\times g$", midpoint(A--B), A+B, tg);
		label("$\times g$", midpoint(B--C), B+C, tg);
		label("$\times g$", midpoint(C--A), C+A, tg);
		label("$3$", origin, origin, pointpen);
		add(shift( (-3.2,0.1) ) * CC());

		label("$g = \operatorname{Frob}_{\mathfrak P}$", (-1.7,1.7), origin, tg);

		pointpen = heavygreen;
		pair W = Drawing("g_4H", dir(50), dir(50), pointpen);
		pair X = Drawing("g_5H", W*dir(90), W*dir(90), pointpen);
		pair Y = Drawing("g_6H", W*dir(180), W*dir(180), pointpen);
		pair Z = Drawing("g_7H", W*dir(270), W*dir(270), pointpen);
		draw(W--X, dashed + pointpen, EndArrow, Margin(2,2));
		draw(X--Y, dashed + pointpen, EndArrow, Margin(2,2));
		draw(Y--Z, dashed + pointpen, EndArrow, Margin(2,2));
		draw(Z--W, dashed + pointpen, EndArrow, Margin(2,2));
		defaultpen(red);
		label("$\times g$", W--X, W+X, tg);
		label("$\times g$", X--Y, X+Y, tg);
		label("$\times g$", Y--Z, Y+Z, tg);
		label("$\times g$", Z--W, Z+W, tg);
		label("$4$", origin, origin, pointpen);

		label("$\boxed{n = 7 = 3+4}$", (-2,-1.8), origin, black);
	\end{asy}
\end{center}

The theorem is that these coincide:
\begin{theorem}
	[Frobenius elements control decomposition]
	\label{thm:frob_control_decomp}
	Let $\QQ \subseteq E \subseteq K$ an extension of number fields
	and assume $K/\QQ$ is Galois (though $E/\QQ$ need not be).
	Pick an unramified rational prime $p$; let $G = \Gal(K/\QQ)$
	and $H$ the corresponding intermediate subgroup.
	Finally, let $\kP$ be a prime above $p$ in $K$.

	Then the \emph{factorization pattern} of $p$ in $E$ is given by
	the \emph{cycle structure} of $\Frob_{\kP}$ acting on the left cosets of $H$.
\end{theorem}
Often, we take $E = K$, in which case this is just asserting
that the decomposition of the prime $p$ is controlled by a Frobenius element over it.

An important special case is when $E = \QQ(\alpha)$,
because as we will see it is let us determine how the minimal
polynomial of $\alpha$ factors modulo $p$.
To motivate this, let's go back a few chapters
and think about the Factoring Algorithm.

Let $\alpha$ be an algebraic integer and $f$ its minimal polynomial (of degree $n$).
Set $E = \QQ(\alpha)$ (which has degree $n$ over $\QQ$).
Suppose we're lucky enough that $\OO_E = \ZZ[\alpha]$,
i.e.\ that $E$ is monogenic.
Then we know by the Factoring Algorithm,
to factor any $p$ in $E$, all we have to do is factor $f$ modulo $p$,
since if $f = f_1^{e_1} \dots f_g^{e_g} \pmod p$ then we have
\[ (p) = \prod_i \kp_i = \prod_i (f_i(\alpha), p)^{e_i}. \]
This gives us complete information about the ramification indices and inertial degrees;
the $e_i$ are the ramification indices, and $\deg f_i$ are the inertial degrees
(since $\OO_E / \kp_i \cong \FF_p[X] / (f_i(X))$).

In particular, if $p$ is unramified then all the $e_i$ are equal to $1$, and we get
\[ n = \deg f = \deg f_1 + \deg f_2 + \dots + \deg f_g. \]
Once again we have a partition of $n$;
we call this the \vocab{factorization pattern} of $f$ modulo $p$.
So, to see the factorization pattern of an unramified $p$ in $\OO_E$,
we just have to know the factorization pattern of the $f \pmod p$.

Turning this on its head, if we want to know the factorization pattern of $f \pmod p$,
we just need to know how $p$ decomposes.
And it turns out these coincide even without the assumption that $E$ is monogenic.

\begin{theorem}[Frobenius controls polynomial factorization]
	\label{thm:factor_poly_frob}
	Let $\alpha$ be an algebraic integer with minimal polynomial $f$,
	and let $E = \QQ(\alpha)$.
	Then for any prime $p$ unramified in the splitting field $K$ of $f$,
	the following coincide:
	\begin{enumerate}[(i)]
		\ii The factorization pattern of $p$ in $E$.
		\ii The factorization pattern of $f \pmod p$.
		\ii The cycle structure associated to the action
		of $\Frob_{\kP} \in \Gal(K/\QQ)$ on the roots of $f$,
		where $\kP$ is above $p$ in $K$.
	\end{enumerate}
\end{theorem}
\begin{example}[Factoring $x^3-2 \pmod 5$]
	Let $\alpha = \cbrt2$ and $f = x^3-2$, so $E = \QQ(\cbrt2)$.
	Set $p=5$ and finally, let $K = \QQ(\cbrt2, \omega)$ be the splitting field.
	Setup:
	\begin{center}
	\begin{tikzcd}
		K = \QQ(\cbrt2, \omega) \ar[d, dash, "2"']
			& \kP \ar[d, dash]
			& x^3-2 = (x-\cbrt2)(x-\cbrt2\omega)(x-\cbrt2\omega^2) \\
		E = \QQ(\sqrt[3]{2}) \ar[d, dash, "3"']
			& \kp \ar[d, dash]
			& x^3-2 = (x-\cbrt2)(x^2+\cbrt2x+\cbrt4) \\
		\QQ & (5) & x^3-2 \text{ irreducible over } \QQ
	\end{tikzcd}
	\end{center}
	The three claimed objects now all have shape $2+1$:
	\begin{enumerate}[(i)]
		\ii By the Factoring Algorithm, we have
		$(5) = (5, \cbrt2-3)(5, 9+3\cbrt2+\cbrt4)$.
		\ii We have $x^3-2 \equiv (x-3)(x^2+3x+9) \pmod 5$.
		\ii We saw before that $\Frob_{\kP} = (\bullet)(\bullet \; \bullet)$.
	\end{enumerate}
\end{example}

\begin{proof}[Sketch of Proof]
	Letting $n = \deg f$.
	Let $H$ be the subgroup of $G = \Gal(K/\QQ)$ corresponding to $E$, so $[G:E] = n$.
	Pictorially, we have
	\begin{center}
	\begin{tikzcd}
		K \ar[d, dash] & \{1\} \ar[d, dash] & \kP \ar[d, dash] \\
		E = \QQ(\alpha) \ar[d, dash] & H \ar[d, dash] & \kp \ar[d, dash] \\
		\QQ & G & (p)
	\end{tikzcd}
	\end{center}
	We claim that (i), (ii), (iii) are all equivalent to
	\begin{center}
		(iv) The pattern of the action of $\Frob_{\kP}$ on the $G/H$.
	\end{center}
	In other words we claim the cosets correspond to the $n$ roots of $f$ in $K$.
	Indeed $H$ is just the set of $\tau \in G$ such that $\tau(\alpha)=\alpha$,
	so there's a bijection between the roots and the cosets $G/H$
	by $\tau H \mapsto \tau(\alpha)$.
	Think of it this way: if $G = S_n$, and $H = \{\tau : \tau(1) = 1\}$,
	then $G/H$ has order $n! / (n-1)! = n$ and corresponds to the elements $\{1, \dots, n\}$.
	So there is a natural bijection from (iii) to (iv).

	The fact that (i) is in bijection to (iv) was the previous theorem,
	\Cref{thm:frob_control_decomp}.
	The correspondence (i) $\iff$ (ii) is a fact of Galois theory,
	so we omit the proof here.
\end{proof}

All this can be done in general with $\QQ$ replaced by $F$;
for example, in \cite{ref:lenstra_chebotarev}.

\section{Example application: IMO 2003 problem 6}
As an example of the power we now have at our disposal, let's prove:

\begin{center}
	\begin{minipage}{4.5cm}
		\includegraphics[width=4cm]{media/IMO-2003-logo.png}
	\end{minipage}%
	\begin{minipage}{10cm}
		\textbf{Problem 6}.
		Let $p$ be a prime number.
		Prove that there exists a prime number $q$ such that for every integer $n$,
		the number $n^p-p$ is not divisible by $q$.
	\end{minipage}
\end{center}
We will show, much more strongly, that there exist infinitely many primes $q$
such that $X^p-p$ is irreducible modulo $q$.

\begin{proof}[Solution]
	Okay! First, we draw the tower of fields
	\[ \QQ \subseteq \QQ(\sqrt[p]{p}) \subseteq K \]
	where $K$ is the splitting field of $f(x) = x^p-p$.
	Let $E = \QQ(\sqrt[p]{p})$ for brevity and note it has degree $[E:\QQ] = p$.
	Let $G = \Gal(K/\QQ)$.
	\begin{ques}
		Show that $p$ divides the order of $G$. (Look at $E$.)
	\end{ques}
	Hence by Cauchy's theorem (\Cref{thm:cauchy_group}, which is a purely group-theoretic fact)
	we can find a $\sigma \in G$ of order $p$.
	By Chebotarev, there exist infinitely many rational (unramified) primes $q \neq p$
	and primes $\kQ \subseteq \OO_K$ above $q$
	such that $\Frob_\kQ = \sigma$.
	(Yes, that's an uppercase Gothic $Q$. Sorry.)

	We claim that all these $q$ work.

	By \Cref{thm:factor_poly_frob}, the factorization of $f \pmod q$ is
	controlled by the action of $\sigma = \Frob_\kQ$ on the roots of $f$.
	But $\sigma$ has prime order $p$ in $G$!
	So all the lengths in the cycle structure have to divide $p$.
	Thus the possible factorization patterns of $f$ are
	\[ p = \underbrace{1 + 1 + \dots + 1}_{\text{$p$ times}}
	\quad\text{or}\quad p = p. \]
	So we just need to rule out the $p = 1 + \dots + 1$ case now:
	this only happens if $f$ breaks into linear factors mod $q$.
	Intuitively this edge case seems highly unlikely (are we really so unlucky
	that $f$ factors into \emph{linear} factors when we want it to be irreducible?).
	And indeed this is easy to see: this means that $\sigma$ fixes all
	of the roots of $f$ in $K$, but that means $\sigma$ fixes $K$ altogether,
	and hence is the identity of $G$, contradiction.
\end{proof}
\begin{remark}
	In fact $K = \QQ(\sqrt[p]{p}, \zeta_p)$, and $\left\lvert G \right\rvert = p(p-1)$.
	With a little more group theory, we can show that in fact the density of
	primes $q$ that work is $\frac 1p$.
\end{remark}

\section\problemhead

\begin{problem}
	Show that for an odd prime $p$, \[ \left( \frac 2p \right) = (-1)^{\frac 18(p^2-1)}. \]
	\begin{hint}
		Modify the end of the proof of quadratic reciprocity.
	\end{hint}
	\begin{sol}
		It is still true that
		\[ \left( \frac 2q \right) = 1
		\iff \sigma_2 \in H \iff \text{$2$ splits in $\ZZ\left[ \tfrac12(1+\sqrt{q^\ast}) \right]$}. \]
		Now, $2$ splits in the ring if and only if $t^2 - t - \tfrac14(1-q^\ast)$
		factors mod $2$. This happens if and only if $q^\ast \equiv 1 \pmod 8$.
		One can check this is exactly if $q \equiv \pm 1 \pmod 8$, which gives the conclusion.
	\end{sol}
\end{problem}

\begin{problem}
	Let $f$ be a nonconstant polynomial with integer coefficients.
	Suppose $f \pmod p$ splits completely into linear factors
	for all sufficiently large primes $p$.
	Show that $f$ splits completely into linear factors.
\end{problem}

\begin{dproblem}
	[Dirichlet's theorem on arithmetic progressions]
	Let $a$ and $m$ be relatively prime positive integers.
	Show that the density of primes $p \equiv a \pmod m$ is exactly $\frac{1}{\phi(m)}$.
	\begin{hint}
		Chebotarev Density on $\QQ(\zeta_m)$.
	\end{hint}
	\begin{sol}
		Let $K = \Gal(\QQ(\zeta_m)/\QQ)$.
		One can show that $\Gal(K/\QQ) \cong \Zm m$ exactly as before.
		In particular, $\Gal(K/\QQ)$ is abelian and therefore its conjugacy classes
		are singleton sets; there are $\phi(m)$ of them.

		As long as $p$ is sufficiently large, it is unramified
		and $\sigma_p = \Frob_\kp$ for any $\kp$ above $p$
		(as $m$th roots of unity will be distinct modulo $p$;
		differentiate $x^m-1$ mod $p$ again).
	\end{sol}
\end{dproblem}

\begin{problem}
	Let $n$ be an odd integer which is not a prime power.
	Show that the $n$th cyclotomic polynomial is not
	irreducible modulo \emph{any} rational prime.
	% https://mathoverflow.net/questions/12366/how-many-primes-stay-inert-in-a-finite-non-cyclic-extension-of-number-fields
\end{problem}

\begin{problem}
	[Putnam 2012 B6]
	\yod
	Let $p$ be an odd prime such that $p \equiv 2 \pmod 3$.
	Let $\pi$ be a permutation of $\FF_p$ by $\pi(x) = x^3 \pmod p$.
	Show that $\pi$ is even if and only if $p \equiv 3 \pmod 4$.
	\begin{hint}
		By primitive roots, it's the same as the action of $\times 3$ on $\Zcc{p-1}$.
		Let $\zeta$ be a $(p-1)$st root of unity.
		Take $d = \prod_{i < j} (\zeta^i - \zeta^j)$, think about $\QQ(d)$,
		and figure out how to act on it by $x \mapsto x^3$.
	\end{hint}
	\begin{sol}
		This solution is by David Corwin.
		By primitive roots, it's the same as the action of $\times 3$ on $\Zcc{p-1}$.
		Let $\zeta$ be a $(p-1)$st root of unity.

		Consider
		\[ d = \prod_{0 \le i < j < p-1} (\zeta^i - \zeta^j). \]
		This is the square root of the discriminant of
		the polynomial $X^{p-1}-1$; in other words $d^2 \in \ZZ$.
		In fact, by elementary methods one can compute
		\[ (-1)^{\binom{p-1}{2}} d^2 = -(p-1)^{p-1} \]

		Now take the extension $K = \QQ(d)$, noting that
		\begin{itemize}
			\ii If $p \equiv 3 \pmod 4$, then $d = (p-1)^{\half(p-1)}$, so $K = \QQ$.
			\ii If $p \equiv 1 \pmod 4$, then $d = i(p-1)^{\half(p-1)}$, so $K = \QQ(i)$.
		\end{itemize}
		Either way, in $\OO_K$, let $\kp$ be a prime ideal above $(3) \subseteq \OO_K$.
		Let $\sigma = \Frob_\kp$ then be the unique element such that
		$\sigma(x) = x^3 \pmod{\kp}$ for all $x$.
		Then, we observe that
		\[
			\sigma(d) \equiv \prod_{0 \le i < j < p-1} (\zeta^{3i} - \zeta^{3j})
			\equiv \begin{cases}
				+d & \text{if $\pi$ is even} \\
				-d & \text{if $\pi$ is odd}
			\end{cases} \pmod{\kp}.
		\]
		Now if $K = \QQ$, then $\sigma$ is the identity, thus $\sigma$ even.
		Conversely, if $K = \QQ(i)$, then $3$ does not split, so $\sigma(d) = -d$
		(actually $\sigma$ is complex conjugation) thus $\pi$ is odd.

		Note the condition that $p \equiv 2 \pmod 3$ is used only
		to guarantee that $\pi$ is actually a permutation (and thus $d \neq 0$);
		it does not play any substantial role in the solution.
	\end{sol}
\end{problem}
