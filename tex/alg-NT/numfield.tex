\chapter{Algebraic integers}
\label{ch:algebraic_integers}
Here's a first taste of algebraic number theory.

This is really close to the border between olympiads and higher math.
You've always known that $a+\sqrt2 b$ had a ``norm'' $a^2-2b^2$,
and that somehow this norm was multiplicative.
You've also always known that roots come in conjugate pairs.
You might have heard of minimal polynomials but not know much about them.

This chapter and the next one will make all these vague notions precise.
It's drawn largely from the first chapter of \cite{ref:oggier_NT}.

\section{Motivation from high school algebra}
This is adapted from my blog, \emph{Power Overwhelming}\footnote{URL:
\url{https://blog.evanchen.cc/2014/10/19/why-do-roots-come-in-conjugate-pairs/}}.

In high school precalculus, you'll often be asked to find the roots of some polynomial with integer coefficients.
For instance,
\[ x^3 - x^2 - x - 15 = (x-3)(x^2+2x+5) \]
has roots $3$, $-1+2i$, $-1-2i$.
Or as another example,
\[ x^3 - 3x^2 - 2x + 2 = (x+1)(x^2-4x+2) \]
has roots $-1$, $2 + \sqrt 2$, $2 - \sqrt 2$.
You'll notice that the irrational roots, like $-1 \pm 2i$ and $2 \pm \sqrt 2$, are coming up in pairs. In fact, I think precalculus explicitly tells you that the complex roots come in conjugate pairs. More generally, it seems like all the roots of the form $a + b \sqrt c$ come in ``conjugate pairs''. And you can see why.

But a polynomial like
\[ x^3 - 8x + 4 \]
has no rational roots.
(The roots of this are approximately $-3.0514$, $0.51730$, $2.5341$.)
Or even simpler,
\[ x^3 - 2 \]
has only one real root, $\sqrt[3]{2}$.
These roots, even though they are irrational, have no ``conjugate'' pairs.
Or do they?

Let's try and figure out exactly what's happening.
Let $\alpha$ be any complex number.
We define a \vocab{minimal polynomial} of $\alpha$ over $\QQ$ to be a polynomial such that
\begin{itemize}
	\item $P(x)$ has rational coefficients, and leading coefficient $1$,
	\item $P(\alpha) = 0$.
	\item The degree of $P$ is as small as possible.
		We call $\deg P$ the \vocab{degree} of $\alpha$.
\end{itemize}
\begin{example}[Examples of minimal polynomials]
	\listhack
	\begin{enumerate}[(a)]
	\ii $\sqrt 2$ has minimal polynomial $x^2-2$.
	\ii The imaginary unit $i = \sqrt{-1}$ has minimal polynomial $x^2+1$.
	\ii A primitive $p$th root of unity, $\zeta_p = e^{\frac{2\pi i}{p}}$, has minimal polynomial $x^{p-1} + x^{p-2} + \dots + 1$, where $p$ is a prime.
	\end{enumerate}
\end{example}
Note that $100x^2 - 200$ is also a polynomial of the same degree which has $\sqrt 2$ as a root; that's why we want to require the polynomial to be monic. That's also why we choose to work in the rational numbers; that way, we can divide by leading coefficients without worrying if we get non-integers.

Why do we care? The point is as follows: suppose we have another polynomial $A(x)$ such that $A(\alpha) = 0$.
Then we claim that $P(x)$ actually divides $A(x)$!
That means that all the other roots of $P$ will also be roots of $A$.

The proof is by contradiction: if not, by polynomial long division we can find a quotient and remainder $Q(x)$, $R(x)$ such that
\[ A(x) = Q(x) P(x) + R(x) \]
and $R(x) \not\equiv 0$.
Notice that by plugging in $x = \alpha$, we find that $R(\alpha) = 0$.
But $\deg R < \deg P$, and $P(x)$ was supposed to be the minimal polynomial.
That's impossible!

It follows from this and the monotonicity of the minimal polynomial
that it is unique (when it exists), so actually it is better to refer to
\emph{the} minimal polynomial.
\begin{exercise}
	Can you find an element in $\CC$
	that has no minimal polynomial?
\end{exercise}

Let's look at a more concrete example.
Consider $A(x) = x^3-3x^2-2x+2$ from the beginning.
The minimal polynomial of $2 + \sqrt 2$ is $P(x) = x^2 - 4x + 2$ (why?).
Now we know that if $2 + \sqrt 2$ is a root, then $A(x)$ is divisible by $P(x)$.
And that's how we know that if $2 + \sqrt 2$ is a root of $A$, then $2 - \sqrt 2$ must be a root too.

As another example, the minimal polynomial of $\sqrt[3]{2}$ is $x^3-2$. So $\sqrt[3]{2}$ actually has \textbf{two} conjugates, namely, $\alpha = \sqrt[3]{2} \left( \cos 120^\circ + i \sin 120^\circ \right)$ and $\beta = \sqrt[3]{2} \left( \cos 240^\circ + i \sin 240^\circ \right)$. Thus any polynomial which vanishes at $\sqrt[3]{2}$ also has $\alpha$ and $\beta$ as roots!

\begin{ques}
	[Important but tautological:
	irreducible $\iff$ minimal]
	Let $\alpha$ be a root of the polynomial $P(x)$.
	Show that $P(x)$ is the minimal polynomial
	if and only if it is irreducible.
	% (This is tautological: the point is just to realize that ``minimal polynomials'' and ``irreducible polynomials'' are the same beasts.)
\end{ques}

\section{Algebraic numbers and algebraic integers}
\prototype{$\sqrt2$ is an algebraic integer (root of $x^2-2$),
$\half$ is an algebraic number but not an algebraic integer (root of $x-\half$).}

Let's now work in much vaster generality.
First, let's give names to the new numbers we've discussed above.
\begin{definition}
	An \vocab{algebraic number} is any $\alpha \in \CC$
	which is the root of \emph{some} polynomial with coefficients in $\QQ$.
	The set of algebraic numbers is denoted $\ol\QQ$.
\end{definition}
\begin{remark}
	One can equally well say algebraic numbers are those that
	are roots of some polynomial with coefficients in $\ZZ$ (rather than $\QQ$),
	since any polynomial in $\QQ[x]$ can be scaled to one in $\ZZ[x]$.
\end{remark}
\begin{definition}
	Consider an algebraic number $\alpha$ and
	its minimal polynomial $P$
	(which is monic and has rational coefficients).
	If it turns out the coefficients of $P$ are integers,
	then we say $\alpha$ is an \vocab{algebraic integer}.

	The set of algebraic integers is denoted $\ol\ZZ$.
\end{definition}
\begin{remark}
	One can show, using \emph{Gauss's Lemma}, that if $\alpha$ is the root
	of \emph{any} monic polynomial with integer coefficients,
	then $\alpha$ is an algebraic integer.
	So in practice, if I want to prove that $\sqrt 2 + \sqrt 3$ is an algebraic integer,
	then I only have to say ``the polynomial $(x^2-5)^2-24$ works''
	without checking that it's minimal.
\end{remark}
Sometimes for clarity, we refer to elements of $\ZZ$
as \vocab{rational integers}.
\begin{example}[Examples of algebraic integers]
	The numbers
	\[ 4, \; i = \sqrt{-1}, \; \sqrt[3]{2}, \; \sqrt2+\sqrt3 \]
	are all algebraic integers, since they are the roots of the monic polynomials
	$x-4$, $x^2+1$, $x^3-2$ and $(x^2-5)^2-24$.

	The number $\half$ has minimal polynomial $x - \half$,
	so it's an algebraic number but not an algebraic integer.
	(In fact, the rational root theorem also directly implies
	that any monic integer polynomial does not have $\half$ as a root!)
\end{example}

There are two properties I want to give
for these off the bat,
because they'll be used extensively in the tricky
(but nice) problems at the end of the section.
The first we prove now, since it's very easy:
\begin{proposition}[Rational algebraic integers are rational integers]
	An algebraic integer is rational
	if and only if it is a rational integer.
	In symbols, \[ \ol\ZZ \cap \QQ = \ZZ. \]
\end{proposition}
\begin{proof}
	Let $\alpha$ be a rational number.
	If $\alpha$ is an integer, it is the root of $x-\alpha$,
	hence an algebraic integer too.

	Conversely, if $P$ is a monic polynomial with integer
	coefficients such that $P(\alpha) = 0$ then
	(by the rational root theorem, say)
	it follows $\alpha$ must be an integer.
\end{proof}
The other is that:
\begin{proposition}
	[$\ol{\ZZ}$ is a ring and $\ol{\QQ}$ is a field]
	The algebraic integers $\ol{\ZZ}$ form a ring.
	The algebraic numbers $\ol{\QQ}$ form a field.
\end{proposition}
We could prove this now if we wanted to,
but the results in the next chapter will more or less
do it for us, and so we take this on faith temporarily.

\section{Number fields}
\prototype{$\QQ(\sqrt2)$ is a typical number field.}

Given any algebraic number $\alpha$,
we're able to consider fields of the form $\QQ(\alpha)$.
Let us write down the more full version.

\begin{definition}
	A \vocab{number field} $K$ is a field containing $\QQ$ as a subfield
	which is a \emph{finite-dimensional} $\QQ$-vector space.
	The \vocab{degree} of $K$ is its dimension.
\end{definition}
\begin{example}[Prototypical example]
	Consider the field
	\[ K = \QQ(\sqrt2) = \left\{ a+b\sqrt2 \mid a,b \in \QQ \right\}. \]
	This is a field extension of $\QQ$,
	and has degree $2$ (the basis being $1$ and $\sqrt2$).
\end{example}

You might be confused that I wrote $\QQ(\sqrt2)$
(which should permit denominators) instead of $\QQ[\sqrt2]$, say.
But if you read through \Cref{ex:gaussian_rationals},
you should see that the denominators don't really matter:
$\frac{1}{3-\sqrt2} = \frac17(3+\sqrt2)$ anyways, for example.
You can either check this now in general,
or just ignore the distinction and pretend I wrote square brackets everywhere.
\begin{exercise}
	[Unimportant]
	Show that if $\alpha$ is an algebraic number,
	then $\QQ(\alpha) \cong \QQ[\alpha]$.
\end{exercise}

\begin{example}[Adjoining an algebraic number]
	Let $\alpha$ be the root of some irreducible polynomial $P(x) \in \QQ[x]$.
	The field $\QQ(\alpha)$ is a field extension as well, and the basis
	is $1, \alpha, \alpha^2, \dots, \alpha^{m-1}$,
	where $m$ is the degree of $\alpha$.
	In particular, the degree of $\QQ(\alpha)$ is just the degree of $\alpha$.
\end{example}
\begin{example}[Non-examples of number fields]
	$\RR$ and $\CC$ are not number fields since there is no \emph{finite}
	$\QQ$-basis of them.
\end{example}

\section{Primitive element theorem, and monogenic extensions}
\prototype{$\QQ(\sqrt3,\sqrt5) \cong \QQ(\sqrt3+\sqrt5)$. Can you see why?}

I'm only putting this theorem here because I was upset that no one
told me it was true (it's a very natural conjecture),
and I hope to not do the same to the reader.
However, I'm not going to use it in anything that follows.

\begin{theorem}
	[Artin's primitive element theorem]
	Every number field $K$ is isomorphic to $\QQ(\alpha)$
	for some algebraic number $\alpha$.
	\label{thm:artin_primitive_elm}
\end{theorem}
The proof is left as \Cref{prob:artin_primitive_elm}, since to prove it I need to talk
about field extensions first.

The prototypical example \[ \QQ(\sqrt3,\sqrt5) \cong \QQ(\sqrt3+\sqrt5) \]
makes it clear why this theorem should not be too surprising.
%To see why this is true, note that $K$ contains the element
%\[ \left( \sqrt3+\sqrt5 \right)^2-8 = 2\sqrt15 \]
%and hence also the element $2(\sqrt15)(\sqrt3+\sqrt5) = 6\sqrt5+10\sqrt3$.
%Thus from the fact that $6\sqrt5+10\sqrt3$ and $\sqrt3+\sqrt5$ are in $K$,
%we can extract both $\sqrt 3$ and $\sqrt 5$.
%Thus $\sqrt3, \sqrt5 \in K$, which is what we wanted.

\section{\problemhead}

\begin{problem}
	Find a polynomial with integer coefficients
	which has $\sqrt2+\sqrt[3]{3}$ as a root.
\end{problem}

\begin{problem}
	[Brazil 2006]
	\gim
	Let $p$ be an irreducible polynomial in $\QQ[x]$
	and degree larger than $1$.
	Prove that if $p$ has two roots $r$ and $s$ whose product is $1$
	then the degree of $p$ is even.
	\begin{hint}
		Note that $p(x)$ is a minimal polynomial for $r$,
		but so is $q(x) = x^{\deg p} p(1/x)$.
		So $q$ and $p$ must be multiples of each other.
	\end{hint}
\end{problem}

\begin{sproblem}
	\label{prob:rep_lemma}
	Consider $n$ roots of unity $\eps_1$, \dots, $\eps_n$.
	Assume the average $\frac1n(\eps_1 + \dots + \eps_n)$ is an algebraic integer.
	Prove that either the average is zero or $\eps_1 = \dots = \eps_n$.
	(Used in \Cref{lem:burnside_ant_lemma}.)
	\begin{hint}
		$\left\lvert \frac 1n(\eps_1 + \dots + \eps_n) \right\rvert \le 1$.
	\end{hint}
\end{sproblem}

\begin{dproblem}
	\gim
	Which rational numbers $q$ satisfy $\cos(q\pi) \in \QQ$?
	\begin{hint}
		Only the obvious ones.
		Assume $\cos(q\pi) \in \QQ$.
		Let $\zeta$ be a root of unity (algebraic integer
		as $\zeta^N-1 = 0$ for some $N$)
		and note that $2\cos(q\pi) = \zeta + \zeta^{N-1}$
		is both an algebraic integer and a rational number.
	\end{hint}
\end{dproblem}

\begin{problem}
	[MOP 2010]
	There are $n > 2$ lamps arranged in a circle;
	initially one is on and the others are off.
	We may select any regular polygon whose vertices are among the lamps
	and toggle the states of all the lamps simultaneously.
	Show it is impossible to turn all lamps off.
	\begin{hint}
		View as roots of unity. Note $\half$ isn't an algebraic integer.
	\end{hint}
\end{problem}

\begin{problem}[Kronecker's theorem]
	\yod
	Let $\alpha$ be an algebraic integer.
	Suppose all its Galois conjugates have absolute value one.
	Prove that $\alpha^N=1$ for some positive integer $N$.
	\begin{hint}
		Let $\alpha = \alpha_1$, $\alpha_2$, \dots, $\alpha_n$ be its conjugates.
		Look at the polynomial $(x-\alpha_1^e) \dots (x-\alpha_n^e)$ across $e \in \NN$.
		Pigeonhole principle on all possible polynomials.
	\end{hint}
\end{problem}

\begin{problem}
	\yod
	Is there an algebraic integer with absolute value one
	which is not a root of unity?
\end{problem}

\begin{problem}
	Is the ring of algebraic integers Noetherian?
\end{problem}
