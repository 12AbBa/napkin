\chapter{Power series and Taylor series}
Polynomials are very well-behaved functions,
and are studied extensively for that reason.
From an analytic perspective, for example, they are smooth,
and their derivatives are easy to compute.

In this chapter we will study \emph{power series},
which are literally ``infinite polynomials'' $\sum_n a_n x^n$.
Armed with our understanding of series and differentiation,
we will see three great things:
\begin{itemize}
	\ii Many of the functions we see in nature
	actually \emph{are} given by power series.
	Among them are $e^x$, $\log x$, $\sin x$.

	\ii Their convergence properties are actually quite well behaved:
	from the string of coefficients,
	we can figure out which $x$ they converge for.

	\ii The derivative of $\sum_n a_n x^n$
	is actually just $\sum_n n a_n x^{n-1}$.
\end{itemize}

\section{Motivation}
To get the ball rolling, let's start with
one infinite polynomial you'll recognize:
for any fixed number $-1 < x < 1$ we have the series convergence
\[ \frac{1}{1-x} = 1 + x + x^2 + \dots \]
by the geometric series formula.

Let's pretend we didn't see this already in
\Cref{prob:geometric}.
So, we instead have a smooth function $f \colon (-1,1) \to \RR$ by
\[ f(x) = \frac{1}{1-x}. \]
Suppose we wanted to pretend that it was equal to
an ``infinite polynomial'' near the origin, that is
\[ (1-x)\inv = a_0 + a_1 x + a_2 x^2 + a_3 x^3 + a_4 x^4 + \dots. \]
How could we find that polynomial,
if we didn't already know?

Well, for starters we can first note that by plugging in $x = 0$
we obviously want $a_0 =1$.

We have derivatives, so actually,
we can then differentiate both sides to obtain that
\[ (1-x)^{-2} = a_1 + 2a_2 x + 3a_3 x^2 + 4a_4 x^3 + \cdots. \]
If we now set $x = 0$, we get $a_1 = 1$.
In fact, let's keep taking derivatives and see what we get.
\begin{alignat*}{8}
	(1-x)^{-1} &{}={}& a_0 &{}+{}& a_1x &{}+{}& a_2x^2 &{}+{}& a_3x^3 &{}+{}& a_4x^4  &{}+{}& a_5x^5 &{}+{}& \dots \\
	(1-x)^{-2} &{}={}& && a_1 &{}+{}& 2a_2 x &{}+{}& 3a_3 x^2 &{}+{}& 4a_4 x^3 &{}+{}& 5a_5 x^4 &{}+{}&\dots \\
	2(1-x)^{-3} &{}={}& && && 2a_2 &{}+{}& 6a_3 x &{}+{}& 12 a_4 x^2 &{}+{}& 20 a_5 x^3 &{}+{}&  \dots \\
	6(1-x)^{-4} &{}={}& &&  &&&& 6a_3 &{}+{}& 24 a_4 x &{}+{}& 60 a_5 x^2 &{}+{}& \dots \\
	24(1-x)^{-5} &{}={}& && && && && 24 a_4 &{}+{}& 120 a_5 x &{}+{}& \dots \\
	&{}\vdotswithin={}.
\end{alignat*}
If we set $x=0$ we find $1 = a_0 = a_1 = a_2 = \dots$
which is what we expect;
the geometric series $\frac{1}{1-x} = 1 + x + x^2 + \dots$.
And so actually taking derivatives was enough to get the right claim!

\section{Power series}
\prototype{$\frac{1}{1-z} = 1 + z + z^2 + \dots$, which converges on $(-1,1)$.}
Of course this is not rigorous,
since we haven't described what the right-hand side is,
much less show that it can be differentiated term by term.
So we define the main character now.

\begin{definition}
	A \vocab{power series} is a sum of the form
	\[ \sum_{n = 0}^\infty a_n z^n
		= a_0 + a_1 z + a_2 z^2 + \dots \]
	where $a_0$, $a_1$, \dots\ are real numbers,
	and $z$ is a variable.
\end{definition}
\begin{abuse}
	[$0^0=1$]
	If you are very careful, you might notice
	that when $z=0$ and $n=0$ we find $0^0$ terms appearing.
	For this chapter the convention is that
	they are all equal to one.
\end{abuse}

Now, if I plug in a \emph{particular} real number $h$,
then I get a series of real numbers $\sum_{n = 0}^{\infty} a_n h^n$.
So I can ask, when does this series converge?
It terms out there is a precise answer for this.

\begin{definition}
	Given a power series $\sum_{n=0}^{\infty} a_n z^n$,
	the \vocab{radius of convergence} $R$ is defined
	by the formula
	\[ \frac 1R = \limsup_{n \to \infty} \left\lvert a_n \right\rvert^{1/n}. \]
	with the convention that $R = 0$ if the right-hand side is $\infty$,
	and $R = \infty$ if the right-hand side is zero.
\end{definition}
\begin{theorem}
	[Cauchy-Hadamard theorem]
	Let $\sum_{n=0}^{\infty} a_n z^n$
	be a power series with radius of convergence $R$.
	Let $h$ be a real number, and consider the infinite series
	\[ \sum_{n=0}^\infty a_n h^n \]
	of real numbers.
	Then:
	\begin{itemize}
		\ii The series converges absolutely if $|h| < R$.
		\ii The series diverges if $|h| > R$.
	\end{itemize}
\end{theorem}
\begin{proof}
	This is not actually hard,
	but it won't be essential, so not included.
\end{proof}
\begin{remark}
	In the case $|h| = R$, it could go either way.
\end{remark}
\begin{example}
	[$\sum z^n$ has radius $1$]
	Consider the geometric series $\sum_{n} z^n = 1 + z + z^2 + \dots$.
	Since $a_n = 1$ for every $n$, we get $R = 1$,
	which is what we expected.
\end{example}

Therefore, if $\sum_n a_n z^n$ is a power
series with a nonzero radius $R > 0$ of convergence,
then it can \emph{also} be thought of as a function
\[ (-R, R) \to \RR
	\quad\text{ by }\quad
	h \mapsto \sum_{n \ge 0} a_n h^n. \]
This is great.
Note also that if $R = \infty$,
this means we get a function $\RR \to \RR$.

\begin{abuse}
	[Power series vs.\ functions]
	There is some subtlety going on with ``types'' of objects again.
	Analogies with polynomials can help.

	Consider $P(x) = x^3 + 7x + 9$, a polynomial.
	You \emph{can}, for any real number $h$,
	plug in $P(h)$ to get a real number.
	However, in the polynomial \emph{itself},
	the symbol $x$ is supposed to be a \emph{variable} ---
	which sometimes we will plug in a real number for,
	but that happens only after the polynomial is defined.

	Despite this, ``the polynomial $p(x) = x^3+7x+9$''
	(which can be thought of as the coefficients)
	and ``the real-valued function $x \mapsto x^3+7x+9$''
	are often used interchangeably.
	The same is about to happen with power series:
	while they were initially thought of as a sequence of
	coefficients, the Cauchy-Hadamard theorem
	lets us think of them as functions too,
	and thus we blur the distinction between them.
%	Pedants will sometimes \emph{define} a polynomial
%	to be the sequence of coefficients, say $(9,7,0,1)$,
%	with $x^3+7x+9$ being ``intuition only''.
%	Similarly they would define a power series
%	to be the sequence $(a_0, a_1, \dots)$.
%	We will not go quite this far, but they have a point.
%
%	I will be careful to use ``power series''
%	for the one with a variable,
%	and ``infinite series'' for the sums of real numbers from before.
\end{abuse}

\section{Differentiating them}
\prototype{We saw earlier $1+x+x^2+x^3+\dots$ has derivative $1+2x+3x^2+\dots$.}
As promised, differentiation works exactly as you want.

\begin{theorem}
	[Differentiation works term by term]
	Let $\sum_{n \ge 0} a_n z^n$ be a power series
	with radius of convergence $R > 0$,
	and consider the corresponding function
	\[ f \colon (-R,R) \to \RR \quad\text{by}\quad
		f(x) = \sum_{n \ge 0} a_n x^n. \]
	Then all the derivatives of $f$ exist and
	are given by power series
	\begin{align*}
		f'(x) &= \sum_{n \ge 1} n a_n x^{n-1} \\
		f''(x) &= \sum_{n \ge 2} n(n-1) a_n x^{n-2} \\
		&\vdotswithin=
	\end{align*}
	which also converge for any $x \in (-R, R)$.
	In particular, $f$ is smooth.
\end{theorem}
\begin{proof}
	Also omitted.
	The right way to prove it is to define the notion ``converges uniformly'',
	and strengthen Cauchy-Hadamard to have this as a conclusion as well.
	However, we won't use this later.
\end{proof}

\begin{corollary}
	[A description of power series coefficients]
	Let $\sum_{n \ge 0} a_n z^n$ be a power series
	with radius of convergence $R > 0$,
	and consider the corresponding function $f(x)$ as above.
	Then
	\[ a_n = \frac{f^{(n)}(x)}{n!}. \]
\end{corollary}
\begin{proof}
	Take the $n$th derivative and plug in $x=0$.
\end{proof}

\section{Analytic functions}
\prototype{The piecewise $e^{-1/x}$ or $0$ function is not analytic,
	but is smooth.}
With all these nice results about power series,
we now have a way to do this process the other way:
suppose that $f \colon U \to \RR$ is a function.
Can we express it as a power series?

Functions for which this \emph{is} true
are called analytic.
\begin{definition}
	A function $f \colon U \to \RR$ is \vocab{analytic} at
	the point $p \in U$
	if there exists an open neighborhood $V$ of $p$ (inside $U$)
	and a power series $\sum_n a_n z^n$ such that
	\[ f(x) = \sum_{n \ge 0} a_n (x-p)^n \]
	for any $x \in V$.
	As usual, the whole function is analytic
	if it is analytic at each point.
\end{definition}
\begin{ques}
	Show that if $f$ is analytic, then it's smooth.
\end{ques}
Moreover, if $f$ is analytic,
then by the corollary above its coefficients are
actually described exactly by
\[ f(x) = \sum_{n \ge 0} \frac{f^{(n)}(p)}{n!} (x-p)^n. \]
Even if $f$ is smooth but not analytic,
we can at least write down the power series;
we give this a name.
\begin{definition}
	For smooth $f$,
	the power series $\sum_{n \ge 0} \frac{f^{(n)}(p)}{n!} z^n$
	is called the \vocab{Taylor series} of $f$ at $p$.
\end{definition}

\begin{example}
	[Examples of analytic functions]
	\listhack
	\label{ex:nonanalytic}
	\begin{enumerate}[(a)]
		\ii Polynomials, $\sin$, $\cos$, $e^x$, $\log$
		all turn out to be analytic.

		\ii The smooth function from before defined by
		\[ f(x) = \begin{cases}
				\exp(-1/x) & x > 0 \\
				0 & x \le 0 \\
			\end{cases}
		\]
		is \emph{not} analytic.
		Indeed, suppose for contradiction it was.
		As all the derivatives are zero,
		its Taylor series would be $0 + 0x + 0x^2 + \dots$.
		This Taylor series does \emph{converge}, but not to the right value ---
		as $f(\eps) > 0$ for any $\eps > 0$, contradiction.
	\end{enumerate}
\end{example}

\begin{theorem}
	[Analytic iff Taylor series has positive radius]
	Let $f \colon U \to \RR$ be a smooth function.
	Then $f$ is analytic if and only if for any point $p \in U$,
	its Taylor series at $p$ has positive radius of convergence.
\end{theorem}

\begin{example}
	It now follows that $f(x) = \sin(x)$ is analytic.
	To see that, we can compute
	\begin{align*}
		f(0) &= \sin 0 = 0 \\
		f'(0) &= \cos 0 = 1 \\
		f''(0) &= -\sin 0 = 0 \\
		f^{(3)}(0) &= -\cos 0 = -1 \\
		f^{(4)}(0) &= \sin 0 = 0 \\
		f^{(5)}(0) &= \cos 0 = 1 \\
		f^{(6)}(0) &= -\sin 0 = 0 \\
		&\vdotswithin=
	\end{align*}
	and so by continuing the pattern
	(which repeats every four) we find the Taylor series is
	\[ z - \frac{z^3}{3!} + \frac{z^5}{5!} - \frac{z^7}{7!} + \dots \]
	which is seen to have radius of convergence $R = \infty$.
\end{example}

Like with differentiable functions:
\begin{proposition}
	[All your usual closure properties for analytic functions]
	The sums, products, compositions, nonzero quotients
	of analytic functions are analytic.
\end{proposition}
The upshot of this is that most of your usual
functions that occur in nature,
or even artificial ones like $f(x) = e^x + x \sin(x^2)$,
will be analytic, hence describable locally by Taylor series.

\section{A definition of Euler's constant and exponentiation}
We can actually give a definition of $e^x$ using the tools we have now.

\begin{definition}
	We define the map $\exp \colon \RR \to \RR$ by
	using the following power series,
	which has infinite radius of convergence:
	\[ \exp(x) = \sum_{n \ge 0} \frac{x^n}{n!}. \]
	We then define Euler's constant as $e = \exp(1)$.
\end{definition}
\begin{ques}
	Show that under this definition, $\exp' = \exp$.
\end{ques}

We are then settled with:
\begin{proposition}
	[$\exp$ is multiplicative]
	Under this definition,
	\[ \exp(x+y) = \exp(x) \exp(y). \]
\end{proposition}
\begin{proof}
	[Idea of proof.]
	There is some subtlety here with switching
	the order of summation that we won't address.
	Modulo that:
	\begin{align*}
		\exp(x) \exp(y)
		&= \sum_{n \ge 0} \frac{x^n}{n!}
			\sum_{m \ge 0} \frac{y^m}{m!}
		= \sum_{n \ge 0} \sum_{m \ge 0}
			\frac{x^n}{n!} \frac{y^m}{m!} \\
		&= \sum_{k \ge 0} \sum_{\substack{m+n = k \\ m,n \ge 0}}
			\frac{x^n y^m}{n! m!}
		= \sum_{k \ge 0} \sum_{\substack{m+n = k \\ m,n \ge 0}}
			\binom{k}{n} \frac{x^n y^m}{k!} \\
		&= \sum_{k \ge 0} \frac{(x+y)^k}{k!} = \exp(x+y). \qedhere
	\end{align*}
\end{proof}

\begin{corollary}
	[$\exp$ is positive]
	\listhack
	\begin{enumerate}[(a)]
		\ii We have $\exp(x) > 0$ for any real number $x$.
		\ii The function $\exp$ is strictly increasing.
	\end{enumerate}
\end{corollary}
\begin{proof}
	First \[ \exp(x) = \exp(x/2)^2 \ge 0 \]
	which shows $\exp$ is nonnegative.
	Also, $1 = \exp(0) = \exp(x) \exp(-x)$ implies $\exp(x) \neq 0$
	for any $x$, proving (a).

	(b) is just since $\exp'$ is strictly positive (racetrack principle).
\end{proof}

The log function then comes after.
\begin{definition}
	We may define $\log \colon \RR_{>0} \to \RR$
	to be the inverse function of $\exp$.
\end{definition}
Since its derivative is $1/x$ it is smooth;
and then one may compute its coefficients to show it is analytic.

Note that this actually gives us a rigorous way to define
$a^r$ for any $a > 0$ and $r > 0$, namely
\[ a^r \defeq \exp\left( r \log a \right). \]

\section{This all works over complex numbers as well,
except also complex analysis is heaven}
We now mention that every theorem we referred to above
holds equally well if we work over $\CC$,
with essentially no modifications.
\begin{itemize}
	\ii Power series are defined by $\sum_n a_n z^n$ with $a_n \in \CC$,
	rather than $a_n \in \RR$.
	\ii The definition of radius of convergence $R$ is unchanged!
	The series will converge if $|z| < R$.
	\ii Differentiation still works great.
	(The definition of the derivative is unchanged.)
	\ii Analytic still works great for functions
	$f \colon U \to \CC$, with $U \subseteq \CC$ open.
\end{itemize}
In particular, we can now even define complex exponentials,
giving us a function \[ \exp \colon \CC \to \CC \]
since the power series still has $R = \infty$.
More generally if $a > 0$ and $z \in \CC$
we may still define \[ a^z \defeq \exp(z \log a). \]
(We still require the base $a$ to be a positive real
so that $\log a$ is defined, though.
So this $i^i$ issue is still there.)

However, if one tries to study calculus for complex functions
as we did for the real case,
in addition to most results carrying over,
we run into a huge surprise:
\begin{quote}
	\itshape
	If $f \colon \CC \to \CC$ is differentiable,
	it is analytic.
\end{quote}
And this is just the beginning of the nearly unbelievable
results that hold for complex analytic functions.
But this is the part on real analysis,
so you will have to read about this later!

\section{\problemhead}

\begin{problem}
	Find the Taylor series of $\log(1-x)$.
\end{problem}

\begin{dproblem}
	[Euler formula]
	Show that \[ \exp(i \theta) = \cos \theta + i \sin \theta \]
	for any real number $\theta$.
	\begin{hint}
		Because you know all derivatives of $\sin$ and $\cos$,
		you can compute their Taylor series,
		which converge everywhere on $\RR$.
		At the same time, $\exp$ was defined as a Taylor series,
		so you can also compute it.
		Write them all out and compare.
	\end{hint}
\end{dproblem}

\begin{dproblem}
	[Taylor's theorem, Lagrange form]
	Let $f \colon [a,b] \to \RR$ be continuous
	and $n+1$ times differentiable on $(a,b)$.
	Define
	\[ P_n = \sum_{k=0}^n \frac{f^{(k)}(a)}{k!} \cdot (b-a)^k. \]
	Prove that there exists $\xi \in (a,b)$ such that
	\[ f^{(n+1)}(\xi) = (n+1)! \cdot \frac{f(b) - P_n}{(b-a)^{n+1}}. \]
	This generalizes the mean value theorem
	(which is the special case $n = 0$, where $P_0 = f(a)$).
	\begin{hint}
		Use repeated Rolle's theorem.
		You don't need any of the theory in this chapter to solve this,
		so it could have been stated much earlier;
		but then it would be quite unmotivated.
	\end{hint}
\end{dproblem}

\begin{problem}
	[Putnam 2018 A5]
	\yod
	Let $f \colon \RR \to \RR$ be smooth,
	and assume that $f(0) = 0$, $f(1) = 1$, and $f(x) \ge 0$
	for every real number $x$.
	Prove that $f^{(n)}(x) < 0$ for some positive integer $n$
	and real number $x$.
	\begin{hint}
		Use Taylor's theorem.
	\end{hint}
\end{problem}
