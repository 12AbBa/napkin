\chapter[Bonus: A hint of p-adic numbers]{Bonus: A hint of $p$-adic numbers}
This is a bonus chapter meant for those
who have also read about \textbf{rings and fields}:
it's a nice tidbit at the intersection of algebra and analysis.


In this chapter, we are going to redo most of the previous chapter
with the absolute value $\left\lvert - \right\rvert$
replaced by the $p$-adic one.
This will give us the $p$-adic integers $\ZZ_p$,
and the $p$-adic numbers $\QQ_p$.
The one-sentence description is that these are
``integers/rationals carrying full mod $p^e$ information''
(and only that information).

In everything that follows $p$ is always assumed to denote a prime.
The first four sections will cover the founding definitions
culminating in a short solution to a USA TST problem.
We will then state (mostly without proof) some more
surprising results about continuous functions $f \colon \ZZ_p \to \QQ_p$;
finally we close with the famous proof of the Skolem-Mahler-Lech theorem
using $p$-adic analysis.

\section{Motivation}
Before really telling you what $\ZZ_p$ and $\QQ_p$ are,
let me tell you what you might expect them to do.

In elementary/olympiad number theory,
we're already well-familiar with the following two ideas:
\begin{itemize}
	\ii Taking modulo a prime $p$ or prime power $p^e$, and
	\ii Looking at the exponent $\nu_p$.
\end{itemize}

Let me expand on the first point.
Suppose we have some Diophantine equation.
In olympiad contexts, one can take an equation modulo $p$
to gain something else to work with.
Unfortunately, taking modulo $p$ loses some information:
the reduction $\ZZ \surjto \ZZ/p$ is far from injective.

If we want finer control, we could consider instead
taking modulo $p^2$, rather than taking modulo $p$.
This can also give some new information (cubes modulo $9$, anyone?),
but it has the disadvantage that $\ZZ/p^2$ isn't a field,
so we lose a lot of the nice algebraic properties that we
got if we take modulo $p$.

One of the goals of $p$-adic numbers is that we can get around
these two issues I described.
The $p$-adic numbers we introduce is going to have the following properties:
\begin{enumerate}
	\ii \textbf{You can ``take modulo $p^e$ for all $e$ at once''.}
	In olympiad contexts, we are used to picking a particular modulus and then
	seeing what happens if we take that modulus.
	But with $p$-adic numbers, we won't have to make that choice.
	An equation of $p$-adic numbers carries enough information
	to take modulo $p^e$.
	\ii \textbf{The numbers $\QQ_p$ form a field},
	the nicest possible algebraic structure:
	$1/p$ makes sense.
	Contrast this with $\ZZ/p^2$, which is not even an integral domain.
	\ii \textbf{It doesn't lose as much information}
	as taking modulo $p$ does:
	rather than the surjective $\ZZ \surjto \ZZ/p$ we have an
	\emph{injective} map $\ZZ \injto \ZZ_p$.
	\ii \textbf{Despite this, you ``ignore'' some ``irrelevant'' data}.
	Just like taking modulo $p$, you want to zoom-in on
	a particular type of algebraic information,
	and this means necessarily
	losing sight of other things.\footnote{To draw an analogy: the equation
	$ a^2 + b^2 + c^2 + d^2 = -1$
	has no integer solutions, because, well, squares are nonnegative.
	But you will find that this equation has solutions modulo any prime $p$,
	because once you take modulo $p$ you stop being able to
	talk about numbers being nonnegative.
	The same thing will happen if we work in $p$-adics:
	the above equation has a solution in $\ZZ_p$ for every prime $p$.}
\end{enumerate}
So, you can think of $p$-adic numbers as the right tool to use
if you only really care about modulo $p^e$ information,
but normal $\ZZ/p^e$ isn't quite powerful enough.

To be more concrete, I'll give a poster example now:
\begin{example}
	[USA TST 2002/2]
	For a prime $p$, show the value of
	\[ f_p(x) = \sum_{k=1}^{p-1} \frac{1}{(px+k)^2} \pmod{p^3} \]
	does not depend on $x$.
	\label{ex:token}
\end{example}
Here is a problem where we \emph{clearly} only care
about $p^e$-type information.
Yet it's a nontrivial challenge to do the
necessary manipulations mod $p^3$ (try it!).
The basic issue is that there is no good way to deal with
the denominators modulo $p^3$
(in part $\ZZ/p^3$ is not even an integral domain).

However, with $p$-adic analysis we're going to be able
to overcome these limitations and give a ``straightforward'' proof
by using the identity
\[
	\left( 1 + \frac{px}{k} \right)^{-2}
	= \sum_{n \ge 0} \binom{-2}{n} \left( \frac{px}{k} \right)^n.
\]
Such an identity makes no sense over $\QQ$ or $\RR$
for convergence reasons,
but it will work fine over the $\QQ_p$, which is all we need.


\section{Algebraic perspective}
\prototype{$-1/2 = 1 + 3 + 3^2 + 3^3 + \dots \in \ZZ_3$.}
We now construct $\ZZ_p$ and $\QQ_p$.
I promised earlier that a $p$-adic integer will let you look at
``all residues modulo $p^e$'' at once.
This definition will formalize this.

\subsection{Definition of $\ZZ_p$}
\begin{definition}
	[Introducing $\ZZ_p$]
	A \vocab{$p$-adic integer} is a sequence
	\[ x = (x_1 \bmod p, \; x_2 \bmod{p^2}, \; x_3 \bmod{p^3}, \; \dots) \]
	of residues $x_e$ modulo $p^e$ for each integer $e$,
	satisfying the compatibility relations
	$x_i \equiv x_j \pmod{p^i}$ for $i < j$.

	The set $\ZZ_p$ of $p$-adic integers forms a ring
	under component-wise addition and multiplication.
\end{definition}

\begin{example}
	[Some $3$-adic integers]
	Let $p=3$.
	Every usual integer $n$ generates
	a (compatible) sequence of residues modulo $p^e$ for each $e$,
	so we can view each ordinary integer as $p$-adic one:
	\[ 50 = \left( 2 \bmod 3, \; 5 \bmod 9, \;
		23 \bmod{27}, \; 50 \bmod{81}, \; 50 \bmod{243}, \; \dots \right). \]
	On the other hand, there are sequences of residues
	which do not correspond to any usual integer despite
	satisfying compatibility relations, such as
	\[ \left( 1 \bmod 3, \; 4 \bmod 9, \;
		13 \bmod{27}, \; 40 \bmod{81}, \; \dots \right) \]
	which can be thought of as $x = 1 + p + p^2 + \dots$.
\end{example}
In this way we get an injective map
\[ \ZZ \injto \ZZ_p \qquad
	n \mapsto \left( n \bmod p, n \bmod{p^2}, n \bmod{p^3}, \dots \right) \]
which is not surjective.
So there are more $p$-adic integers than usual integers.

(Remark for experts:
those of you familiar with category theory might recognize
that this definition can be written concisely as
\[ \ZZ_p \defeq \varprojlim \ZZ/p^e \ZZ \]
where the inverse limit is taken across $e \ge 1$.)

\begin{exercise}
	Check that $\ZZ_p$ is an integral domain.
\end{exercise}

\subsection{Base $p$ expansion}
Here is another way to think about $p$-adic integers using ``base $p$''.
As in the example earlier, every usual integer can be written in base $p$,
for example
\[ 50 = \ol{1212}_3 = 2 \cdot 3^0 + 1 \cdot 3^1 + 2 \cdot 3^2 + 1 \cdot 3^3. \]
More generally, given any $x = (x_1, \dots) \in \ZZ_p$,
we can write down a ``base $p$'' expansion in the sense
that there are exactly $p$ choices of $x_k$ given $x_{k-1}$.
Continuing the example earlier, we would write
\begin{align*}
	\left( 1 \bmod 3, \; 4 \bmod 9, \;
	13 \bmod{27}, \; 40 \bmod{81}, \; \dots \right)
	&= 1 + 3 + 3^2 + \dots \\
	&= \ol{\dots1111}_3
\end{align*}
and in general we can write
\[ x = \sum_{k \ge 0} a_k p^k = \ol{\dots a_2 a_1 a_0}_p \]
where $a_k \in \{0, \dots, p-1\}$,
such that the equation holds modulo $p^e$ for each $e$.
Note the expansion is infinite to the \emph{left},
which is different from what you're used to.

(Amusingly, negative integers also have infinite base $p$ expansions:
$-4 = \ol{\dots222212}_3$, corresponding to
$(2 \bmod 3, \; 5 \bmod 9, \; 23 \bmod{27}, \; 77 \bmod{81} \dots)$.)

Thus you may often hear the advertisement that a $p$-adic integer
is an ``possibly infinite base $p$ expansion''.
This is correct, but later on we'll be thinking of $\ZZ_p$ in
a more and more ``analytic'' way,
and so I prefer to think of this as
\begin{moral}
	$p$-adic integers are Taylor series with base $p$.
\end{moral}
Indeed, much of your intuition from generating functions $K[[X]]$
(where $K$ is a field) will carry over to $\ZZ_p$.

\subsection{Constructing $\QQ_p$}
Here is one way in which your intuition from generating functions carries over:
\begin{proposition}
	[Non-multiples of $p$ are all invertible]
	The number $x \in \ZZ_p$ is invertible if and only if $x_1 \ne 0$.
	In symbols,
	\[ x \in \ZZ_p^\times \iff x \not\equiv 0 \pmod p. \]
\end{proposition}
Contrast this with the corresponding statement for $K[ [ X ] ]$:
a generating function $F \in K[ [ X ] ]$ is invertible iff $F(0) \neq 0$.

\begin{proof}
	If $x \equiv 0 \pmod p$ then $x_1 = 0$,
	so clearly not invertible.
	Otherwise, $x_e \not\equiv 0 \pmod p$ for all $e$,
	so we can take an inverse $y_e$ modulo $p^e$,
	with $x_e y_e \equiv 1 \pmod{p^e}$.
	As the $y_e$ are themselves compatible,
	the element $(y_1, y_2, \dots)$ is an inverse.
\end{proof}
\begin{example}
	[We have $-\half = \ol{\dots1111}_3 \in \ZZ_3$]
	We claim the earlier example is actually
	\begin{align*}
	-\half = \left( 1 \bmod 3, \; 4 \bmod 9, \;
	13 \bmod{27}, \; 40 \bmod{81}, \; \dots \right)
	&= 1 + 3 + 3^2 + \dots \\
	&= \ol{\dots1111}_3.
	\end{align*}
	Indeed, multiplying it by $-2$ gives
	\[ \left( -2 \bmod 3, \; -8 \bmod 9, \;
		-26 \bmod{27}, \; -80 \bmod{81}, \; \dots \right)
		= 1. \]
	(Compare this with the ``geometric series''
	$1 + 3 + 3^2 + \dots = \frac{1}{1-3}$.
	We'll actually be able to formalize this later, but not yet.)
\end{example}
\begin{remark}
	[$\half$ is an integer for $p > 2$]
	The earlier proposition implies that $\half \in \ZZ_3$
	(among other things);
	your intuition about what is an ``integer'' is different here!
	In olympiad terms, we already knew $\half \pmod 3$ made sense,
	which is why calling $\half$ an ``integer''
	in the $3$-adics is correct,
	even though it doesn't correspond to any element of $\ZZ$.
\end{remark}
\begin{exercise}
	[Unimportant but tricky]
	Rational numbers correspond exactly to
	eventually periodic base $p$ expansions.
\end{exercise}

With this observation, here is now the definition of $\QQ_p$.
\begin{definition}
	[Introducing $\QQ_p$]
	Since $\ZZ_p$ is an integral domain,
	we let $\QQ_p$ denote its field of fractions.
	These are the \vocab{$p$-adic numbers}.
\end{definition}
Continuing our generating functions analogy:
\[ \ZZ_p \text{ is to } \QQ_p
	\quad\text{as}\quad
	K[[X]] \text{ is to } K((X)). \]
This means
\begin{moral}
	$\QQ_p$ can be thought of as Laurent series with base $p$.
\end{moral}
and in particular according to the earlier proposition we deduce:
\begin{proposition}
	[$\QQ_p$ looks like formal Laurent series]
	Every nonzero element of $\QQ_p$ is uniquely of the form
	\[ p^k u \qquad \text{ where } k \in \ZZ, \; u \in \ZZ_p^\times. \]
\end{proposition}
Thus, continuing our base $p$ analogy,
elements of $\QQ_p$ are in bijection with ``Laurent series''
\[ \sum_{k \ge -n} a_k p^k
	= \ol{\dots a_2 a_1 a_0 . a_{-1} a_{-2} \dots a_{-n}}_p \]
for $a_k \in \left\{ 0, \dots, p-1 \right\}$.
So the base $p$ representations of elements of $\QQ_p$
can be thought of as the same as usual,
but extending infinitely far to the left
(rather than to the right).

\begin{remark}
	[Warning]
	The field $\QQ_p$ has characteristic \emph{zero}, not $p$.
\end{remark}
\begin{remark}
	[Warning on fraction field]
	This result implies that you shouldn't think about elements of $\QQ_p$
	as $x/y$ (for $x,y \in \ZZ_p$) in practice,
	even though this is the official definition
	(and what you'd expect from the name $\QQ_p$).
	The only denominators you need are powers of $p$.

	To keep pushing the formal Laurent series analogy,
	$K((X))$ is usually not thought of as quotient of generating functions
	but rather as ``formal series with some negative exponents''.
	You should apply the same intuition on $\QQ_p$.
\end{remark}

\begin{remark}
At this point I want to make a remark about the fact $1/p \in \QQ_p$,
connecting it to the wish-list of properties I had before.
In elementary number theory you can take equations modulo $p$,
but if you do the quantity $n/p \bmod{p}$ doesn't make sense
unless you know $n \bmod{p^2}$.
You can't fix this by just taking modulo $p^2$
since then you need $n \bmod{p^3}$ to get $n/p \bmod{p^2}$, ad infinitum.
You can work around issues like this,
but the nice feature of $\ZZ_p$ and $\QQ_p$
is that you have modulo $p^e$ information for ``all $e$ at once'':
the information of $x \in \QQ_p$ packages all the modulo $p^e$
information simultaneously.
So you can divide by $p$ with no repercussions.
\end{remark}


\section{Analytic perspective}
\subsection{Definition}
Up until now we've been thinking about things mostly algebraically,
but moving forward it will be helpful to start using the language of analysis.
Usually, two real numbers are considered ``close'' if
they are close on the number of line,
but for $p$-adic purposes we only care about modulo $p^e$ information.
So, we'll instead think of two elements of $\ZZ_p$ or $\QQ_p$
as ``close'' if they differ by a large multiple of $p^e$.

For this we'll borrow the familiar $\nu_p$ from elementary number theory.
\begin{definition}
	[$p$-adic valuation and absolute value]
	We define the \vocab{$p$-adic valuation} $\nu_p : \QQ_p^\times \to \ZZ$
	in the following two equivalent ways:
	\begin{itemize}
		\ii For $x = (x_1, x_2, \dots) \in \ZZ_p$ we let
		$\nu_p(x)$ be the largest $e$ such that
		$x_e \equiv 0 \pmod{p^e}$ (or $e=0$ if $x \in \ZZ_p^\times$).
		Then extend to all of $\QQ_p^\times$
		by $\nu_p(xy) = \nu_p(x) + \nu_p(y)$.
		\ii Each $x \in \QQ_p^\times$ can be written
		uniquely as $p^k u$ for $u \in \ZZ_p^\times$, $k \in \ZZ$.
		We let $\nu_p(x) = k$.
	\end{itemize}
	By convention we set $\nu_p(0) = +\infty$.
	Finally, define the \vocab{$p$-adic absolute value}
	$\left\lvert \bullet \right\rvert_p$ by
	\[ \left\lvert x \right\rvert_p = p^{-\nu_p(x)}. \]
	In particular $\left\lvert 0 \right\rvert_p = 0$.
\end{definition}
This fulfills the promise that $x$ and $y$ are close
if they look the same modulo $p^e$ for large $e$;
in that case $\nu_p(x-y)$ is large
and accordingly $\left\lvert x-y \right\rvert_p$ is small.

\subsection{Ultrametric space}
In this way, $\QQ_p$ and $\ZZ_p$ becomes a metric space
with metric given by $\left\lvert x-y \right\rvert_p$.

\begin{exercise}
	\label{exer:alternating}
	Suppose $f \colon \ZZ_p \to \QQ_p$ is continuous
	and $f(n) = (-1)^n$ for every $n \in \ZZ_{\ge 0}$.
	Prove that $p = 2$.
\end{exercise}

In fact, these spaces satisfy a stronger
form of the triangle inequality than you are used to from $\RR$.
\begin{proposition}
	[$\left\lvert \bullet \right\rvert_p$ is an ultrametric]
	For any $x,y \in \ZZ_p$, we have the \vocab{strong triangle inequality}
	\[ \left\lvert x+y \right\rvert_p
		\le \max \left\{ \left\lvert x \right\rvert_p,
		\left\lvert y \right\rvert_p \right\}.
	\]
	Equality holds if (but not only if)
	$\left\lvert x \right\rvert_p \neq \left\lvert y \right\rvert_p$.
\end{proposition}

However, $\QQ_p$ is more than just a metric space:
it is a field, with its own addition and multiplication.
This means we can do analysis just like in $\RR$ or $\CC$:
basically, any notion such as ``continuous function'',
``convergent series'', et cetera has a $p$-adic analog.
In particular, we can define what it means for an infinite sum to converge:
\begin{definition}
	[Convergence notions]
	Here are some examples of $p$-adic analogs of ``real-world'' notions.
	\begin{itemize}
		\ii A sequence $s_1$, \dots converges to a limit $L$
		if $\lim_{n \to \infty} \left\lvert s_n - L \right\rvert_p = 0$.
		\ii The infinite series $\sum_k x_k$ converges
		if the sequence of partial sums $s_1 = x_1$,
		$s_2 = x_1 + x_2$, \dots, converges to some limit.
		\ii \dots et cetera \dots
	\end{itemize}
\end{definition}
With this definition in place,
the ``base $p$'' discussion we had earlier is now true
in the analytic sense: if $x = \ol{\dots a_2 a_1 a_0}_p \in \ZZ_p$ then
\[ \sum_{k=0}^\infty a_k p^k \quad\text{converges to } x. \]
Indeed, the difference between $x$ and the $n$th partial sum is divisible by $p^n$,
hence the partial sums approach $x$ as $n \to \infty$.

While the definitions are all the same,
there are some changes in properties that should be true.
For example, in $\QQ_p$ convergence of partial sums is simpler:
\begin{proposition}
	[$|x_k|_p \to 0$ iff convergence of series]
	A series $\sum_{k=1}^\infty x_k$ in $\QQ_p$
	converges to some limit if and only if
	$\lim_{k \to \infty} |x_k|_p = 0$.
	\label{noharmonic}
\end{proposition}
Contrast this with $\sum \frac1n = \infty$ in $\RR$.
You can think of this as a consequence of strong triangle inequality.
\begin{proof}
	By multiplying by a large enough power of $p$,
	we may assume $x_k \in \ZZ_p$.
	(This isn't actually necessary, but makes the notation nicer.)

	Observe that $x_k \pmod p$ must eventually stabilize,
	since for large enough $n$ we have
	$\left\lvert x_n \right\rvert_p < 1 \iff \nu_p(x_n) \ge 1$.
	So let $a_1$ be the eventual residue modulo $p$
	of $\sum_{k=0}^N x_k \pmod p$ for large $N$.
	In the same way let $a_2$ be the eventual residue modulo $p^2$, and so on.
	Then one can check we approach the limit $a = (a_1, a_2, \dots)$.
\end{proof}

\subsection{More fun with geometric series}
Let's finally state the $p$-adic analog of the geometric series formula.

\begin{proposition}
	[Geometric series]
	Let $x \in \ZZ_p$ with $\left\lvert x \right\rvert_p < 1$.
	Then \[ \frac{1}{1-x} = 1 + x + x^2 + x^3 + \dots. \]
\end{proposition}
\begin{proof}
	Note that the partial sums satisfy
	$1 + x + x^2 + \dots + x^n = \frac{1-x^n}{1-x}$,
	and $x^n \to 0$ as $n \to \infty$ since
	$\left\lvert x \right\rvert_p < 1$.
\end{proof}

So, $1 + 3 + 3^2 + \dots = -\half$ is really a correct convergence
in $\ZZ_3$.
And so on.

If you buy the analogy that $\ZZ_p$ is generating functions
with base $p$, then all the olympiad generating functions
you might be used to have $p$-adic analogs.
For example, you can prove more generally that:
\begin{theorem}
	[Generalized binomial theorem]
	If $x \in \ZZ_p$ and $\left\lvert x \right\rvert_p < 1$,
	then for any $r \in \QQ$ we have the series convergence
	\[ \sum_{n \ge 0} \binom rn x^n = (1+x)^r. \]
\end{theorem}
(I haven't defined $(1+x)^r$, but it has the properties you expect.)

\subsection{Completeness}
Note that the definition of $\left\lvert \bullet \right\rvert_p$
could have been given for $\QQ$ as well;
we didn't need $\QQ_p$ to introduce it
(after all, we have $\nu_p$ in olympiads already).
The big important theorem I must state now is:
\begin{theorem}
	[$\QQ_p$ is complete]
	The space $\QQ_p$ is the completion of $\QQ$
	with respect to $\left\lvert \bullet \right\rvert_p$.
\end{theorem}
This is the definition of $\QQ_p$ you'll see more frequently;
one then defines $\ZZ_p$ in terms of $\QQ_p$
(rather than vice-versa) according to
\[ \ZZ_p = \left\{ x \in \QQ_p :
	\left\lvert x \right\rvert_p \le 1 \right\}. \]

\subsection{Philosophical notes}
Let me justify why this definition is philosophically nice.
Suppose you are an ancient Greek mathematician who is given:
\begin{quote}
	\textbf{Problem for Ancient Greeks.}
	Estimate the value of the sum
	\[ S = \frac{1}{1^2} + \frac{1}{2^2} + \dots + \frac{1}{10000^2} \]
	to within $0.001$.
\end{quote}
The sum $S$ consists entirely of rational numbers,
so the problem statement would be fair game for ancient Greece.
But it turns out that in order to get a good estimate,
it \emph{really helps} if you know about the real numbers:
because then you can construct the infinite series
$\sum_{n \ge 1} n^{-2} = \frac16 \pi^2$,
and deduce that $S \approx \frac{\pi^2}{6}$,
up to some small error term from the terms past $\frac{1}{10001^2}$,
which can be bounded.

Of course, in order to have access to enough theory
to prove that $S = \pi^2/6$, you need to have the real numbers;
it's impossible to do calculus in $\QQ$
(the sequence $1$, $1.4$, $1.41$, $1.414$,
is considered ``not convergent''!)

Now fast-forward to 2002, and suppose you are given
\begin{quote}
	\textbf{Problem from USA TST 2002.}
	Estimate the sum
	\[ f_p(x) = \sum_{k=1}^{p-1} \frac{1}{(px+k)^2} \]
	to within mod $p^3$.
\end{quote}
% (i.e.\ to within $p^{-3}$ in $\left\lvert \bullet \right\rvert_p$).
Even though $f_p(x)$ is a rational number,
it still helps to be able to do analysis with infinite sums,
and then bound the error term (i.e.\ take mod $p^3$).
But the space $\QQ$ is not complete with respect
to $\left\lvert \bullet \right\rvert_p$ either,
and thus it makes sense to work in the completion of $\QQ$
with respect to $\left\lvert \bullet \right\rvert_p$.
This is exactly $\QQ_p$.

In any case, let's finally solve \Cref{ex:token}.
\begin{example}
[USA TST 2002]
We will now compute
\[ f_p(x) = \sum_{k=1}^{p-1} \frac{1}{(px+k)^2} \pmod{p^3}. \]
Armed with the generalized binomial theorem,
this becomes straightforward.
\begin{align*}
	f_p(x) &= \sum_{k=1}^{p-1} \frac{1}{(px+k)^2}
	= \sum_{k=1}^{p-1} \frac{1}{k^2}
	\left( 1 + \frac{px}{k} \right)^{-2} \\
	&= \sum_{k=1}^{p-1} \frac{1}{k^2} \sum_{n \ge 0}
	\binom{-2}{n} \left( \frac{px}{k} \right)^{n} \\
	&= \sum_{n \ge 0} \binom{-2}{n}
	\sum_{k=1}^{p-1} \frac{1}{k^2} \left( \frac{x}{k} \right)^{n} p^n \\
	&\equiv \sum_{k=1}^{p-1} \frac{1}{k^2}
	 - 2x \left( \sum_{k=1}^{p-1} \frac{1}{k^3} \right) p
	 + 3x^2 \left( \sum_{k=1}^{p-1} \frac{1}{k^4} \right) p^2 \pmod{p^3}.
\end{align*}
Using the elementary facts that
$p^2 \mid \sum_k k^{-3}$ and $p \mid \sum_k k^{-4}$,
this solves the problem.
\end{example}


%\section{Digression on $\CC_p$}
%Before I go on, I want to mention that $\QQ_p$
%is not algebraically closed.
%So, we can take its algebraic closure $\ol{\QQ_p}$ --- but this
%field is now no longer complete (in the topological sense).
%However, we can then take the completion of this space
%to obtain $\CC_p$.
%In general, completing an algebraically closed field
%remains algebraically closed,
%and so there is a larger space $\CC_p$ which
%is algebraically closed \emph{and} complete.
%This space is called the $p$-adic complex numbers.
%
%We won't need $\CC_p$ at all in what follows,
%so you can forget everything you just read.

\section{Mahler coefficients}
One of the big surprises of $p$-adic analysis is that:
\begin{moral}
	We can basically describe all continuous functions $\ZZ_p \to \QQ_p$.
\end{moral}
They are given by a basis of functions
\[ \binom xn \defeq \frac{x(x-1) \dots (x-(n-1))}{n!} \]
in the following way.
\begin{theorem}
	[Mahler; see {\cite[Theorem 51.1, Exercise 51.b]{ref:schikof}}]
	Let $f \colon \ZZ_p \to \QQ_p$ be continuous, and define
	\begin{equation}
		a_n = \sum_{k=0}^n \binom nk (-1)^{n-k} f(k).
		\label{eq:mahler}
	\end{equation}
	Then $\lim_n a_n = 0$ and \[ f(x) = \sum_{n \ge 0} a_n \binom xn. \]

	Conversely, if $a_n$ is any sequence converging to zero,
	then $f(x) = \sum_{n \ge 0} a_n \binom xn$
	defines a continuous function satisfying \eqref{eq:mahler}.
	% true for C_p too
\end{theorem}
The $a_i$ are called the \emph{Mahler coefficients} of $f$.
\begin{exercise}
	Last post we proved that if $f \colon \ZZ_p \to \QQ_p$ is continuous
	and $f(n) = (-1)^n$ for every $n \in \ZZ_{\ge 0}$ then $p = 2$.
	Re-prove this using Mahler's theorem,
	and this time show conversely that a unique such $f$ exists when $p=2$.
\end{exercise}

You'll note that these are the same finite differences that one
uses on polynomials in high school math contests,
which is why they are also called ``Mahler differences''.
\begin{align*}
	a_0 &= f(0) \\
	a_1 &= f(1) - f(0) \\
	a_2 &= f(2) - 2f(1) + f(0) \\
	a_3 &= f(3) - 3f(2) + 3f(1) - f(0).
\end{align*}
Thus one can think of $a_n \to 0$ as saying that
the values of $f(0)$, $f(1)$, \dots behave like a polynomial modulo $p^e$
for every $e \ge 0$.

The notion ``analytic'' also has a Mahler interpretation.
First, the definition.
\begin{definition}
We say that a function $f \colon \ZZ_p \to \QQ_p$ is \vocab{analytic}
if it has a power series expansion
\[ \sum_{n \ge 0} c_n x^n \quad c_n \in \QQ_p
	\qquad\text{ converging for } x \in \ZZ_p. \]
\end{definition}
\begin{theorem}
	[{\cite[Theorem 54.4]{ref:schikof}}]
	The function $f(x) = \sum_{n \ge 0} a_n \binom xn$ is analytic
	if and only if
	\[ \lim_{n \to \infty} \frac{a_n}{n!} = 0. \]
	% true for Q_p too
\end{theorem}
Analytic functions also satisfy the following niceness result:
\begin{theorem}
	[Strassmann's theorem]
	Let $f \colon \ZZ_p \to \QQ_p$ be analytic.
	Then $f$ has finitely many zeros.
	% should be true for \CC_p? gdi
\end{theorem}

To give an application of these results,
we will prove the following result,
which was interesting even before $p$-adics came along!
\begin{theorem}
	[Skolem-Mahler-Lech]
	Let $(x_i)_{i \ge 0}$ be an integral linear recurrence,
	meaning $(x_i)_{i \ge 0}$ is a sequence of integers
	\[ x_n = c_1 x_{n-1} + c_2 x_{n-2} + \dots + c_k x_{n-k}
		\qquad n = 1, 2, \dots \]
	holds for some choice of integers $c_1$, \dots, $c_k$.
	Then the set of indices $\{ i \mid x_i = 0 \}$
	is eventually periodic.
\end{theorem}

\begin{proof}
	According to the theory of linear recurrences,
	there exists a matrix $A$ such that we can write
	$x_i$ as a dot product
	\[ x_i = \left< A^i u, v \right>. \]
	Let $p$ be a prime not dividing $\det A$.
	Let $T$ be an integer such that $A^T \equiv \id \pmod p$
	(with $\id$ denoting the identity matrix).

	Fix any $0 \le r < N$.
	We will prove that either all the terms
	\[ f(n) = x_{nT+r} \qquad n = 0, 1, \dots \]
	are zero, or at most finitely many of them are.
	This will conclude the proof.

	Let $A^T = \id + pB$ for some integer matrix $B$.
	We have
	\begin{align*}
		f(n) &= \left< A^{nT+r} u, v \right>
		= \left< (\id + pB)^n A^r u, v \right> \\
		&= \sum_{k \ge 0} \binom nk \cdot p^n \left< B^n A^r u, v \right> \\
		&= \sum_{k \ge 0} a_n \binom nk \qquad \text{ where }
			a_n = p^n \left< B^n A^r u, v \right> \in p^n \ZZ.
	\end{align*}
	Thus we have written $f$ in Mahler form.
	Initially, we define $f \colon \ZZ_{\ge 0} \to \ZZ$,
	but by Mahler's theorem (since $\lim_n a_n = 0$)
	it follows that $f$ extends to a function $f \colon \ZZ_p \to \QQ_p$.
	Also, we can check that $\lim_n \frac{a_n}{n!} = 0$
	hence $f$ is even analytic.

	Thus by Strassman's theorem, $f$ is either identically zero,
	or else it has finitely many zeros, as desired.
\end{proof}

\section{\problemhead}
\begin{dproblem}
	[$\ZZ_p$ is compact]
	Show that $\QQ_p$ is not compact, but $\ZZ_p$ is.
	(For the latter, I recommend using sequential continuity.)
\end{dproblem}
\begin{dproblem}
	[Totally disconnected]
	Show that both $\ZZ_p$ and $\QQ_p$ are \emph{totally disconnected}:
	there are no connected sets other than the empty set and singleton sets.
\end{dproblem}

\begin{problem}
	[Mentioned in \href{https://mathoverflow.net/a/81377/70654}{MathOverflow}]
	Let $p$ be a prime.
	Find a sequence $q_1$, $q_2$, \dots of rational numbers such that:
	\begin{itemize}
	\ii the sequence $q_n$ converges to $0$ in the real sense;
	\ii the sequence $q_n$ converges to $2021$ in the $p$-adic sense.
	\end{itemize}
\end{problem}

\begin{problem}
	[USA TST 2011]
	Let $p$ be a prime.
	We say that a sequence of integers $\{z_n\}_{n=0}^\infty$
	is a \emph{$p$-pod} if for each $e \geq 0$,
	there is an $N \geq 0$ such that whenever $m \geq N$,
	$p^e$ divides the sum
	\[ \sum_{k=0}^m (-1)^k \binom mk z_k. \]
	Prove that if both sequences $\{x_n\}_{n=0}^\infty$
	and $\{y_n\}_{n=0}^\infty$ are $p$-pods,
	then the sequence $\{x_n y_n\}_{n=0}^\infty$ is a $p$-pod.
\end{problem}
